\documentclass[12pt]{article}
\usepackage{color,soul,xcolor}
\usepackage{geometry}
\usepackage{graphicx}
\usepackage{apacite}

% to use page refs from the manuscript tex file
\usepackage{xr}
\externaldocument{Heal16implicit}


% Miscellaneous dimensions&
\geometry{letterpaper,left=.75in,right=.75in,top=.75in,bottom=.75in,centering}
\setlength{\parskip}{1ex}
\setlength{\parindent}{0em}
\setlength{\headheight}{15pt}

\bibliographystyle{apacite}

\begin{document}

%%%%%%%%%%%%%%%%%%%%%%%%%%%%%%%%%%%%%%%%%%%%%%%%

%INITIAL SUB EXAMPLE:
Dear Dr. Neath,
 
%Example of how to cite a change on (Page \pageref{TODO-1}) of the manuscript file. In manuscript it should be \label{TODO-1}

I thank you and the reviewers for your comments on my manuscript  ``Temporal Contiguity in Incidentally Encoded Memories'' (JML-17-336). In the manuscript, I presented evidence that the Temporal Contiguity Effect (i.e., that recall of one memory tends to trigger recall of other memories encoded nearby in time) can be dramatically reduced, but not always elmiminated, under incodient encoding conditions. I argued that these findings challenge all existing models of the effect, but espically those that attribute temporal contiguity to deleberate encoding stratgies. In response to the issues raised by yourself and the reviews, I have extensively revised the manuscript. 


% QUOTE: Finally, the most significant issue -- as noted by Reviewer 2 -- is how does Experiment 3 differ from Experiments 1 and 2? The lack of an identifying factor is a severe weakness of the paper. As Reviewer 2 notes, Experiment 3 may be able to serve as an existence proof, but it really offers no additional theoretical insight about when a TCE will be observed and when it won't.
In your action letter you highlighted the lack of theoritical insight on to what factors determine when the TCE will be observed under incidental encoding and when it won't. In an attempt to nail down these factors I conducted and additional experiment.... FILL IN DETAILS of new experiment.

In the course of this it bacame clear that the results of experiment 1, althoguyt qualatativel replciating, were not exactly replicating the outcome of the hypothesis test.  which prompts reworking of discussion and intro of prevuous sections ... This is perhaps a slightly boring conclusion (you need tremendous statistical power), but it is theoritically important because....

Below I have complied a list of the other changes made in response to each point raised in the action letter and the reviews. In the manuscript, I have used\setstcolor{red}  \color{red}red font\color{black}~to highlighted all additions.% and \st{strikethrough} to highlight all deletions.

I want to stress my gratitude to yourself and the reviewers for your very helpful feeback. I hope you agree that the resultsing changes have strenghtend the manuscript considerably.

\vspace{20pt}

\textbf{\large{Action Letter}}


\begin{enumerate}

\item
	% QUOTE: Reviewer 1 comments about them level of detail of the two theoretical accounts, but I'll go further: the introduction is too short. Not everyone is familiar with the TCE and it needs to be described and put into context. Similarly, the two possible explanations are only vaguely described; they need to be described more completely so that the predictions are better understood.
	You echoed Rewiewer 1's request for a more detailed introduction. As requested, the new introduction includes a fuller description of the TCE (Page \pageref{TODO-1}), % just include your standard speal like from the 2014 papers
	% QUOTE:Similarly, the two possible explanations are only vaguely described; they need to be described more completely so that the predictions are better understood.
	an extened discusiion of the two theories and their competing predictions (Page \pageref{TODO-2}),
	% QUOTE: Also missing is a statement separating the encoding conditions (incidental vs. intentional) and the retrieval conditions (direct vs. indirect test). You mention incidental, but the reader has to infer that the critical test conditions will be (direct).
	a clarification that the study involves incidental encoding followed by a \emph{direct} test of recall (Page  \pageref{TODO-3}),
	% QUOTE: you don't describe the Nairne et al. experiment and the reader is therefore unable to know how your study compares to their study.
	and a more detailed description of the Nairne et al. (2017) experiment (Page \pageref{TODO-4}).

\item
	%QUOTE: Reviewer 1 also noted the very small contiguity effect in the explicit conditions relative to other studies. This also needs to be addressed. Is it because you have only 1 list? What if you run a model such that only one list is simulated: does that give smaller effects? Put another way, is it theoretically possible to get a large (> 1) TCE with only 1 list?
	You also highlighted Reviewer 1's observation that the contiguite effect reported here is much smaller than the effect reported in previous studies, even in the explicit encoding conditions. We include a discussion of this small effect size on page \pageref{TODO-5}. In breif, I was able to shed some light on this issue by adding an anlysis of participants second list (see the new Figure on Page \pageref{TODO-6}). These new analyses show that in both incidinaal and explicit conditions the size of the contiguity effect increases considerablly from the first to the second list. By the second list THE EFFECT IS APPROXIMATELY EQUAL TO THAT SEEN IN EXPERIMENTS WITH MANY LISTS SAY SOMETHING ABOUT ASYMETERT TOO! (TRUE??? INCLUDE DIRECT COMPARISION TO PEERS). As I note in the manuscript, this new finding is yet another challenge to most existing models because they have no mechanism to allow contiguity to change dramatically with task experience (as you asked)
	% Perhaps make put a version of this in the paper
	For most models, (e.g., TCM) contiguity would be idential regardless of the number of lists simulated because the effectively erase their memory after each list (as a simplyfing assumption and a practical). Even more recent models (e.g. CMR2) which do simulate multiple lists, would produce roughly equal contiguity from trial-to-trial unless the allowed the parameters which control contiguity to change with task experience.

\item
	% QUOTE: Like Reviewer 1 (see also a comment by Reviewer 2), I wondered if you observed standard-looking serial position functions. These have been reported previously for a single list learned undering incidental instructions (e.g., Neath, 1993, Exp. 1). This may be important if, for example, you get two different shaped functions under explicit and implicit, and get a TCE in the former but not in the latter. Similarly, an output order analyses may be informative. Geoff Ward and colleagues have a number of papers on this. This may also be relevant to the size of the TCE. These sorts of additional analyses may also shed light on why the results of Experiment 3 differs from those of Experiments 1 and 2.
	As requested by yourself and all three Reviewers' the revision includes new figures showing serial position functions for each condition (FIGURES.... \pageref{TODO-7}). I have also include probability of first recall fucntions in these figures. SAY SOMETING ABOUT WHAT, IF ANY, LIGHT THESE SHED ON THE CRPS

\item
	% QUOTE: Reviewer 2 notes that there are a number of studies that evidence for encoding temporal information under incidental learning conditions. Given this, you may wish to rephrase slightly your questions.
	On page (\pageref{TODO-8}) I discuss several studies that suggest temporal information is encoded .... see what these studies actually show. And slightly repharase this studies aims in light of these.
	Neath, I. (1993). Contextual and distinctive processes and the serial position function. Journal of Memory and Language, 32, 820-840. doi:10.1006/jmla.1993.1041

\item
	% QUOTE: While it is laudable to have such a large sample, it is also important to know what the sample is like. What was the mean age? Were they native speakers of English (you are asking them to process and recall English stimuli)? How many were male? How many female? If you did not collect information about this, you need to state that you didn't, and it weakens the results.
	You noted that it is important to know something about the charistics of the sample, such as age and native language, even when the sample size is high. This information is reported for the new Experimet 4 (\pageref{TODO-9}). As I now note on page \pageref{TODO-10}, I did not collect this sort of demographic information for the original 3 studies because of time (and fiscal) constraints: to keep the cost at \$1 per participant (plus Amazon's 40\% surcharge), I needed to keep the task under 10min and adding demographic questionaire would have pushed it over. For the new Exp 4, I dropped the second list from the design to make time for demographic informatiom. I also note that although the dempographics of hte mturk community have been described elsewhere (REFS), lacking information on the particular sample used in Exps 1--3 is a limitation.

\item
	% QUOTE: I have a quibble about a statement on page 11: "the Explicit condition showed a distinctly peaked lag-CRP" I'm not sure I agree with this: For the negative lags in particular, the line looks flat, not peaked. Please provide some statistical test to support this claim. 
	On page \pageref{done-11} I have removed the edited the sentence "...the Explicit condition showed a distinctly peaked lag-CRP Z(TCE) significantly above zero..." to drop drop the claim about peakedness because, as you point out, it was not supported by a statistical test. Now it simply says "...the Explicit condition had a lag-CRP Z(TCE) significantly above zero..."
	% QUOTE: Similarly, I have another quibble with how you treat the correlation between overall level and magnitude of the TEC. In Figure 4, you report that there is no statistical evidence for a correlation, p = 0.586. Yet in the text you talk as if the correlation exists and is significant ("Although the correlation is positive, it is quite small ..."). Again, please either report additional statistical analyses such that you justify talking about a non-significant correlation, or change the text to match the outcome of the statistical test you did report.
	Similarly, on page \pageref{done-12} instead of saying "Although the correlation is positive, it is quite small..." I simply say "The correlation is quite small and non-significant..."

\item
	% QUOTE: At some point, you may wish to mention that incidentally-learned information can be recalled better than intentionally-learned information depending on the processing done at encoding (e.g., Eagle & Leiter, 1964). This has implications for a possible qualification of the statements about overall recall levels.
	On page \pageref{TODO-13} I note that "incidentally-learned information can be recalled better than intentionally-learned information depending on the processing done at encoding (e.g., Eagle \& Leiter, 1964)""
	Eagle, M., \& Leiter, E. (1964). Recall and recognition in intentional and incidental learning. Journal of Experimental Psychology, 68, 58-63.

\item
	% QUOTE: p. 10: Please use colours such as white and light grey (rather than black and dark grey) so that the error bars are visible.
	\pageref{TODO-14} On bar plots with black bars, I have changed the color of the error bars to make them more visiable. 


\end{enumerate}


\vspace{20pt}

\textbf{\large{Reviewer 1}}

\begin{enumerate}

\item 
	% QUOTE: First, it would be nice if differences in the two accounts (controlled vs. automatic) of the temporal contiguity effect (TCE) were more fleshed out. I understand the general idea, but in the current paper the differences seem rather vague. What is it about controlled processes that influences the size of the TCE? Is it strategies or intent? The author suggests that it is not just intent, but there is really no indication of how different strategies would influence it. Similar for the automatic processing account. How/why does this account predict the effect? I have read the prior papers, but much more needs to be said in the current study regarding the two accounts and what the overall theoretical advance is other than current theories are lacking.
	The reviewer requested an elobration of how temporal contiguity is produced under each of the two theoritical accounts of the temporal contiguity effect (controled/strategic vs automatic) and the differences between the proposed mechanisms. The revised introuction includes more details on each theory including examples of how contiguity is claimed to arise under each (Page \pageref{TODO-2}). 

\item
	% QUOTE: Second, one interesting finding in all the experiments was how small the temporal contiguity effect was. Is this because only one list was used? The author and other work by Kahana and collegues has typically found much larger contiguity effects where the lag +1 conditional response probability is closer to .30. Similarly, typically contiguity effects are asymmetrical, but most of the current effects look pretty symmetrical. I'm aware that there are some studies that have found symmetrical effects, but it is not clear what is going on in the current data. Again, much more discussion is needed to better explain what is going on with the current data.
	The reviewered noted that the contiguity effects reported here are both smaller in magnitiude and more symetric than those reported in most other work, and that this is true for both the incidental and explicit encoding conditions. As the reviewer also noted, one possible explination is that that the current study reported data from a signle list. Experiment 1--3 actually presented participants with a second list (as noted in the methods), although the data was not repoprted as it was not relevent to the main questions. But the reviewers commentst. prompted me to include analyses of subject's second list. These analyses show (see the new Figure on Page \pageref{TODO-6}) both incidinaal and explicit conditions the size of the contiguity effect and its asymetery increases considerablly from the first to the second list. By the second list THE EFFECT IS APPROXIMATELY EQUAL TO THAT SEEN IN EXPERIMENTS WITH MANY LISTS (TRUE??? INCLUDE DIRECT COMPARISION TO PEERS). As I note in the manuscript, this new finding is yet another challenge to most existing models because they have no mechanism to allow WHAT DOES IT MEAN THEORITICALLY -- IS THERE ARE PARTICULAR PARAMETER THAT CAN EXPLAIN BOTH THE SIZE AND ASYMETERY??? 

\item
	% QUOTE: Finally, are serial position and probability of first recall curves similar across the different conditions? Examining these other patterns of performance may give some insight into the current results.
	As requested by the reviewer, I have included serial position and probability first recall curves for all of the conditions. THESE SHOW WHAT??? \pageref{TODO-7}



\end{enumerate}


\vspace{20pt}

\textbf{\large{Reviewer 2}}

\begin{enumerate}

	\item 
	% QUOTE: Although not reviewed here, there is already substantial evidence that people encode and retain temporal position and order information under incidental learning conditions (e.g., Nairne, J. S. (1991). Positional uncertainty in long-term memory. Memory & Cognition, 19, 332-340); in this work, temporal contiguity-like effects can be seen in the errors gradients that are produced when people are asked to reconstruct the original orders of presentation. So, the questions are really (1) under what conditions do people use this information strategically to drive recall? and (2) can differential temporal coding, or strategic use of that encoding, be used to explain large and consistent condition differences in recall (such as levels of processing or, in the case of Nairne et al. (2017), the difference between survival processing and controls). The answer to the second question is clearly no, as shown here and in Nairne et al., and the current experiments provide no answer to the first question.
	THIS NEEDS WORK:

	The reviewer pointed out that some previous findings imply that participants encode temporal information under incidental encoding conditions. On page \pageref{TODO-8} of the revision I discuss these findings and recast the questions for the current mansucript. Specifically, the main question for the paper is whether participants will use temporal information to guide free recall (i.e., show a temporal contiguity effect) when presented with a susprize memory test.

	The reviewer suggests that a key question is "under what conditions do people use this information strategically to drive recall?" and his biggest concern is that "it's not clear exactly how the tasks used in Experiment 3, which produce significant TCEs, differ from those used in Experiments 1 and 2". He suggests that progress can be made by holding encoding condistions constant and varying the types of retreival processes encouraged. The new Experiment 4 takes this suggestion by ..... DETAILS..

	THANK NAIRNE FOR SUGGESTING THE RETREIVAL EXP. 

	WHEN TALKIING ABOUT THESE STUDIES MENTION NAIRNES ORDERING TASK!!

	\item
	% QUOTE: I don't know why the task instructions are placed in supplemental materials. They should be in the methods sections.
	I have moved the task instructions from supplemental materials to the methods section.

	\item
	% QUOTE: The discussion of how to calculate the TCE is confusing and will be opaque to most readers. A specific example would help.
	I have added a specific example illusturating the the calculation of the z(TCE) measure  (Page \pageref{TODO-15})

	\item
	% QUOTE: Some readers will probably want conventional statistical analyses which are absent here.
	The reviewer suggested that some readers might prefer convential statistical annlese where werre largeley absent from the manuscript (confidence intervals and shuffling procedures were used instead). As I mentioned above, I have added raw temporal factor scores... DETAILS... 

	And the Experiment 4 includes an ANOVA.

	I have, however, opted to maintain the reliance on confidence intervals rather than t-tests as I find it makes the results section more readable without loosing any information or rigor. 



	

\end{enumerate}





\vspace{20pt}

Thank you for your time and consideration.

\vspace{10pt}

Sincerely,

\vspace{10pt}

Karl Healey\\
khealey@msu.edu\\
Michigan State University\\
Department of Psychology

%\bibliography{healey_lab}

\end{document}



% \vspace{20pt}

% \textbf{\large{Action Letter}}

% We begin by outlining changes we have made in response to several points you raised in addition to the issues raised by the reviewers.

% % QUOTE: For example, much of the current review is based on analyses from the PEERs project. To what extent have these analyses in some form been reported in prior papers? It is critically important that it is made clear what analyses are new (or reanalyses) and what analyses are old.
% \textbf{1.} You asked that we clarify which analyses are completely novel, which are replications of previous findings with new data, and which are reported in existing papers. To make it as easy as possible for the reader to digest this information we have created a new table that lists each analysis, whether the finding is novel, a replication with new data, or a representation of previous work. %TODO: MAKE THIS TABLE


% \textbf{2.} You pointed out that whereas for the PEERS data reported in the manuscript there is a robust contiguity effect regardless of semantic relatedness (Figures 2N and 2O), work by Siddiqui and Unsworth (2011) had found more modest temporal contiguity effects effects when lists were composed of positive, negative, or neutral words. We include a discussion of this finding on page \pageref{AL.2}: "Paste text here!" 

% % QUOTE: Furthermore, it is repeatedly noted that IQ correlates with the temporal factor score. But, to what extent is this relation somewhat spurious given that the IQ is related to recall and recall is related to the temporal factor score? That is, recall ability is the shared source of variance to both. There may be nothing special about the relation between the temporal factor score and IQ scores.
% \textbf{3.} You pointed out that the correlation between temporal factor score and IQ could arise from the fact that overall recall is correlated with both temporal factor and IQ. This is in fact the case, as we have previously shown in Figure 4 of \citeA{HealEtal14}, overall recall scores mediate the correlation between temporal contiguity and IQ. But we maintain that there is something special about temporal contiguity because none of the other measures of recall dynamics we considered (semantic contiguity and measures of how participants recall initiation) correlated with IQ once temporal contiguity is accounted for. In our view, it is an open question whether temporal contiguity per se is special or whether there is some underlying ability (e.g., the ability to constrain search sets, as you have argued), that is critical. In other words, we think the evidence justifies saying temporal contiguity is special, even if it does not justify saying temporal contiguity is casual. On page \pageref{AL.3} we clarify: ``It is important to note that these correlations do not imply causality (it could be that both temporal contiguity and IQ share variance with some third factor) but, critically, semantic clustering does not correlate with IQ. Although the direction of the relationship remains unclear, there is something special about temporal contiguity in predicting recall and IQ.'' %TODO: refine this!

% \vspace{20pt}

% \textbf{\large{Reviewer 1}}

% Reviewer 1 was highly positive and highlighted the fact that the manuscript is ``is timely given the recent Hintzman (2015) article'' and suggested it had ``the potential to provide a highly suitable summary reference from which alternative explanations and viewpoints can depart.''. 

% \vspace{20pt}

% \textbf{\large{Reviewer 2}}

% \textbf{1.} Reviewer 2 indicated that the paper needed to more clearly lay out the key questions it is intended to address. As the reviewer suggested, the manuscript takes aim at several issues that have arisen in the recent literature. On page \pageref{R2.1} we have added a paragraph that clearly specifies these: "FILL IN". and in the discussion we return to these on page \pageref{R2.1b}: "FILL IN" %TODO: Decide what to say here

% \textbf{2.} The Reviewer indicated that our description of Retrieved Context Models did not give the reader an accurate sense of the model's complexity because we omitted descriptions of some parameters. The only major component of the model we did not describe was how it uses the match between each item in its memory and the current state of context to make a decision about which item to recall. In a related point, the reviewer suggests that the model fails to simulate retrieval monitoring processes that ensure participants do not repeatedly recall the same items. While it is true that some models simply ignore repetitions, it is not true of the most recent implementation of retrieved context models, which explicitly model a retrieval monitoring and editing process. We have addressed both the originally un-described retrieval process and the issue of repetitions on page \pageref{R2.2}: ``In addition to context drift, association formation, and context reinstatement, the models also simulate the competitive retrieval process via a evidence accumulation process. More recent versions of the model include mechanisms to screen for intrusions by comparing the context associated with each candidate recall with the current state of context and suppress repetitions by dynamically sets the retrieval threshold of each item as the recall period progresses to allow items that were recalled earlier in the period to participate in, but not dominate, current retrieval competitions \cite{HealKaha15,LohnEtal14}.''

% \textbf{3.} The Reviewer requested that the correction for serial position effects be conducted for all analyses. We have conducted these analyses and added the following description of the results on page \pageref{R2.3}: "FILL IN". %TODO: rerun all analyses including temporal factor correlations
%  Because the correction does not change the key findings, to facilitate comparison between the current work and prior analyses of the contiguity effect, we show the non-corrected results in the main figures and include versions of the figure with the correction as supplemental material.

%  \textbf{4.} The Reviewer argues that to test whether contiguity effects are ``attributable to an automatic process'' or ``to study strategies that subjects use because they know how they are going to be tested.'' we must run an experiment ``in which subjects do not know or suspect that their memory for the words will be tested''. As the reviewer points out, Nairne et al. have run such an experiment, though the results are not yet published. Therefore, in addition to discussing the Nairne result, we have run a new study replicating the finding. A full description of the methods is on page \pageref{R2.4}, but briefly: All participants completed two free recall lists, in the explicit condition they were given standard instructions to study the words and in the implicit condition all mention of memory was omitted. Replicating Nairne et al., we found a clear contiguity effect in the explicit condition, but not the implicit condition. Both groups, showed clear contiguity on a second trial. These results are presented in Figure X and described on \pageref{R2.4b}. On page \pageref{R2.4b} we discuss the theoretical implications of this result: ``FILL IN''. %TODO: what do we want to say about this result?!?

%  %NEXT: finish #5 and #6 from hintzmans review
%  \textbf{5.} The original manuscript included a paragraph that expressed our opinion that data from certain paradigms (e.g., spacing judgments and position judgments) are difficult to interpret as evidence either for or against temporal contiguity because contiguity-generating episodic memory models, like retrieved context models, have not been extended to these paradigms. The Reviewer felt that this position was unwarranted. We disagree and have added the following clarification on Page \pageref{R2.5}: ``Under the models of episodic memory we have discussed, contiguity emerges as a consequence of attempting to recall episodes. These other paradigms add the additional requirement to make judgments about episodes. These judgments are beyond the ken of the existing models. And without clearly specifying the processes that translate into judgments there is simply no principled way to make strong predictions about whether, or how much, contiguity one would expect to see in the data.''

% \textbf{6.} The reviewer requested that we add a discussion of data from two studies by Glenberg \& Bradley. On page \pageref{R2.6} we have added: ``FILL IN''%TODO:read the studies and write someting

% \textbf{7.} Finally, the reviewer indicated that our claim, near the end of the manuscript, that encoding tasks and intent to learn might modulate contiguity effect seemed inconsistent with the earlier sections that argued contiguity was a inherent property of encoding and search processes. The new study showing that contiguity can indeed be eliminated under implicit encoding conditions obviates this concern, in our view. We now acknowledge, early in the manuscript (\pageref{R2.7}, that processes under conscious control, like encoding processes, can reduce and even eliminate contiguity.

% \vspace{20pt}

% \textbf{\large{Reviewer 2}}
% The Reviewer felt that the manuscript was ``a thoughtful response to Hintzman's recent paper in which he questions whether episodic memory is temporally organized'' and saw ``no reason why the authors should not make their case in publication''. The Reviewer did raise some important concerns, which we have attempted to address in the revision.

% \textbf{1.} The reviewers central concern was with our attempt ``to classify different models in order to derive predictions from classes of models instead of considering each model individually''. We agree that there are important differences between models that we have placed in the same class, as well as important similarities among models we have placed in the same class. We have included a new section that acknowledges that the category boundaries are fuzzy on Page \pageref{R3.1}: ``FILL IN.. include something like: This rough categorization has proven to provide a useful framework in the past \cite{FHM,ReviewCh}. Of course, it is intended as a rough heuristic---one that is necessitated by the fact that none of the models have been applied to the range of paradigms. In our view, an important goal for the field as a whole, and proponents of specific models, and the true test lies in extending specific models to simulate the paradigms in question.'' %TODO: write someting that promenently features a nice comment about the malmberg model. Something like this quote from the review: "The Malmberg and Lehman model predicts that the control processes used to encode items plays a crucial role in the degree to which to events are associated. Associative encoding reflects the goals or strategies adopted by the subject, and it is clearly assumed that under some conditions that associative encoding of temporally adjacent events will be actively avoided via compartmentalization"

% \textbf{2.} The reviewer pointed out that there was no reference for the ``Law of Similarity'' quote in the first paragraph. Although we put our statement of the las in quotation, it was actually original text and not quoted from any source. We have removed the quotes.

% \textbf{3.} The Reviewer wondered if Hume's reference to associations was in reference to learning (e.g., conditioning or semantic memory) rather than to episodic memory. Specifically, that ``Our inductive expectation about what the future holds is usually available without reference to any particular event.'' We agree that people make inductions based on many events, not usually single events. But Hume does not make a clear distinction between episodic and non-episodic memory and our interpretation is that he is referring to the abstraction of general rules from the accumulated memory of many individual events. 

% %QUOTE: Since deblurring has never been explained in great detail, with the exception of Goebel and Lewandowsky's (1991) attempt, it is really difficult to make any concrete statements about what effect deblurring would have on retrieval in general and contiguity specifically.
% \textbf{4.}  We suggested that ``in a model in which retrieval is accomplished by correlating a probe with a memory vector (e.g. Lewandowsky \& Murdock, 1989), probing with the recalled item would produce a blurred representation that, on average, is most similar to the item at lag 1 but also somewhat similar to items at remote lags.''. The reviewer argued that the deblurring processes has not been explained in detail (except Goebel and Lewandowsky 1991), it is difficult to make concrete statements. We agree that precise predictions are difficult to make, but it seems quite likely that the process would provide some support for items at remote lags providing a plausible avenue for the models to produce contiguity. %TODO:??? Ask Mike what he thinks

% \textbf{5.} The reviewer pointed out that our description of dual store models omitted some details that have been added to the most recent version, which can modulate the contiguity effect. On page \pageref{R3.5} we have added: ``Current dual-store models have specified some of the control processes that govern how items are processed. For example, Lehman \& Malmberg's (2011) model includes a compartmentalization process that can focus the buffer on particular time periods and can thereby modulate contiguity.'' %TODO: read their description again. 

% \textbf{6.} We have added a reference to Lee and Estes (1977) perturbation models to the position coding section \pageref{R3.6}.

% \textbf{7.} We have removed the sentence "All the models we have considered so far assume that the degree of association between two events is a smoothly decreasing function of the amount of time that separated them." from the hierarchical clustering model section because, as the reviewer points out, some models can use control processes to modulate how items are associated. %TODO: remove the sentence!

% \textbf{8.} We have revised the section on hierarchical clustering models to clarify that ``chunks need not be hierarchically structured under some descriptive label (Malmberg \& Lehman, 2013; Murdock, 1982, 1995).'' %TODO: Write someting about this

% \textbf{9.} In the section on retrieved context models on page \pageref{R3.9} we now note the overlap with some models we have classified as dual store: ``Many of the key features of retrieved context models are shared with other models, especially modern dual-store models, which include similar contextual evolution and reinstatement processes for long-term storage and retrieval \cite{Malmberg,Davelar}''

% \textbf{10.} Early in the original manuscript we argued that one should evaluate a model's based on it ability to account for a wide range of findings. The Reviewer argued that accounting for many data points does not automatically count in a model's favor. Rather one must also consider which data points a model is not able to account for. The reviewer specifically drew attention to the finding that contiguity is reduced when pairs of items are studied (Malmerg \& Lehman, 2013). We have revised this section on page \pageref{R3.10} to make this point: ``FILL IN'' %TODO write someting and include mention of MALMBERG and LEHMAN either here or later and quote both in the letter

% Later in original manuscript we included a section on how to further test the retrieved context framework. We have revised that section to more explicitly point out several predictions of the framework that could be falsified by experiments. This include HIGH MEMORY WIHT NO CONTIGUITY (see page \pageref{R3.10b}).

% \textbf{11.} The Reviewer pointed out that some passages gave the impression that we think contiguity should be all or none under a strategic encoding account. On page \pageref{R3.11} we clarify that ``Of course, different strategies may produce different degrees of contiguity (e.g., a strict serieation strategy that one would apply to memorize a phone number would produce more contiguity than would a strategy based on using the items to tell a story). Moreover, even for a particular strategy, it might generate more or less contiguity depending on how effectively the participant implements it and how well-suited it is to the task. Critically for our purposes, any contiguity effect due to such strategies should be eliminated in situations in which participants are unlikely to employ the strategy. That is, strategy generated contiguity should be much less ubiquitous than contiguity generated by a process integral to memory encoding or search.'' 

% \textbf{12.} In the original manuscript we has suggested ``A strategy-based account has no parameters that map naturally on to age-related change...''. The Reviewer felt that this was overstating the case and pointed to several parameters that might be modulated by age. We agree and have removed this sentence from Page \pageref{R3.12}. %TODO: remove this text from paper

% \textbf{13.} On page \pageref{R3.13} we have revised our discussion of contiguity in continual distractor free recall to clarify that some dual-store models can account for the effect: ``Although a basic dual-store model has difficulty with contiguity in continual distractor free recall because the distractors should empty the buffer and prevent formation of temporally graded associations, modern dual-store models can produce contiguity in the presence of distractors by incorporating contextual drift WHAT DOES MALMBEG MODEL DO \cite{Davelar,Malmberg}'' %TODO: finish this off

% \textbf{14.} The Reviewer argued that is was difficult to put too much weight on the Howard et al. finding of contiguity under extremely fast presentation rates because the data is unavailable. WOULD MARC LET US POST THE DATA??!?? The Reviewer also argued strategic encoding might actually be \emph{more} important under ``difficult'' encoding conditions. We agree that some sources of difficult might force participants to rely more heavily on strategy. But as we argue in the revision on page \pageref{R3.14}, extremely fast presentation rates seem unlikely to do so: ``That is, contiguity was seen when words were presented as rapidly as one every 250 ms. Given evidence that perceptual processing of items takes about 180 ms \cite{REF}, is seems quite implausible to us that participants using the few remaining milliseconds to engage in the sorts of complex strategies that have been suggested to create contiguity (e.g., rehearsal, telling a story).''

% %QUOTE:P.28. On the topic of semantic versus temporal competition, it appears that the asymmetry of the contiguity is disrupted when the semantic associate appears at a longer lag. This is predicted by dual-store models if the prior item was not rehearsed with the subsequent item.
% \textbf{15.} The Reviewer note that in Figure 2 panel O the asymmetry is reduced when an associate is available at a remote lag. %TODO: What to say??!?!?

% \textbf{16.} Regarding our discussion of contiguity in recognition, the Reviewer argued that we have no direct evidence that items were ``recalled''. Unfortunately, there was a typo in the manuscript and we used ``recall'' when me meant ``recognition''. The argument that temporal associations formed between items during study should facilitate recognition (not recall) when list neighbors are probed in succession. This argument holds whether or not participants successfully recollected the first probed item---all that is required is that they successfully recognized it. We have corrected this unfortunate typo on page \pageref{R3.16}.
% %TODO: in ``They found that transitions from recalled items for which the subject "remembered"'' and ``aving two successive probes be from adjacent list positions should help recall''  change RECALLED to endorsed. READ THE WHOLE SECTION CAREFULLY!!

% NEXT: read through, make edits and start making the changes to the text!!


% \vspace{20pt}

% Thank you for your time and consideration.

% \vspace{10pt}

% Sincerely,

% \vspace{10pt}

% Karl Healey\\
% khealey@msu.edu\\
% Michigan State University\\
% Department of Psychology

% \vspace{10pt}

% Michael J. Kahana\\
% kahana@psych.upenn.edu\\
% University of Pennsylvania\\
% Department of Psychology 
 
