\documentclass[12pt]{article}
\usepackage{color,soul,xcolor}
\usepackage{geometry}
\usepackage{graphicx}
\usepackage{apacite}

% to use page refs from the manuscript tex file
\usepackage{xr}
\externaldocument{Heal16implicit}


% Miscellaneous dimensions&
\geometry{letterpaper,left=.75in,right=.75in,top=.75in,bottom=.75in,centering}
\setlength{\parskip}{1ex}
\setlength{\parindent}{0em}
\setlength{\headheight}{15pt}

\bibliographystyle{apacite}

\begin{document}

Dear Dr. Neath,
 
%Example of how to cite a change on (Page \pageref{TODO-1}) of the manuscript file. In manuscript it should be \label{TODO-1}

I thank you and the reviewers for your comments on my manuscript  ``Temporal Contiguity in Incidentally Encoded Memories'' (JML-17-336). The manuscript presented evidence that the Temporal Contiguity Effect (i.e., that recall of one memory tends to trigger recall of other memories encoded nearby in time) can be dramatically reduced, but not always eliminated, under incidental encoding conditions. I argued that these findings challenge all existing models of the effect, but especially those that attribute temporal contiguity to deliberate encoding strategies. In response to the issues raised by yourself and the reviews, I have extensively revised the manuscript. In the manuscript, additions are highlighted with \setstcolor{red}  \color{red}red font\color{black}~ and deletions with \st{strikethrough} (except in the introduction where the edits were too numerous to highlight individually).

% QUOTE: Finally, the most significant issue -- as noted by Reviewer 2 -- is how does Experiment 3 differ from Experiments 1 and 2? The lack of an identifying factor is a severe weakness of the paper. As Reviewer 2 notes, Experiment 3 may be able to serve as an existence proof, but it really offers no additional theoretical insight about when a TCE will be observed and when it won't.
The biggest revision is the addition of new experiment designed to gain insight on when the TCE will be observed under incidental encoding and when it won't (page~\pageref{newexp}). Because the results of this experiment informed many of the other revisions, I describe it first.

The new experiment included a control condition that was almost identical to the ``would the item fit in a Shoebox'' incidental encoding condition of the original Experiment 1 along with two experimental conditions designed to test two different hypotheses. The first hypothesis was that the TCE would is reduced by incidental encoding tasks that require judging each list item against an common referent and the effect could therefore be restored by using a different referent for each item. This hypotheses was tested by modifying the size judgment task to use a different object as the size referent for each list item (i.e, ``is the item larger than a \emph{Unique Referent}''). The second hypothesis was that temporal associations \emph{are} formed under incidental encoding but participants do not spontaneously adopt a retrieval strategy that makes use of them (thanks to Reviewer 2 for suggesting this general direction). This hypothesis was tested by replacing the surprise free recall test with a surprise serial recall test. Because this new Experiment focused on differences among conditions, rather than whether individual conditions were different from zero, I increased the sample size to 500+ in each condition to achieve enough power.

The new data provided no support for the first hypothesis and only marginal evidence for the second. But the control condition turned out to be as informative as the experimental conditions: Although the mean level of the TCE in this incidental encoding control condition was similar the non-significant TCE observed in experiment 1, with the larger sample the standard error was lower and the TCE was significant. This unexpected finding prompted me to revisit the power analysis I had conducted to set sample sizes at the outset of Experiment 1. In that analysis I assumed the implicit TCE effect would be no smaller than one fifth the size of previously reported TCEs and set sample sizes to give 95\% power. But because the single list design increased variance so greatly compared to typical multi-list designs, the observed effect size turned out to be even smaller than expected. A a post hoc power analysis of the first 3 experiments confirmed that experiments 1 and 2, which found null effects, were actually underpowered to detect effects of the size that were found to be significant in the original Experiment 3. I report these new power analyses on page~\pageref{power}.

This discovery does not change the take home message of the paper: that incidental encoding dramatically \emph{reduces} the TCE to levels that challenge existing models. But it does suggest that the strong claim that incidental encoding \emph{eliminates} the TCE is unwarranted. As I note in the discussion ~\pageref{zerovsnear}, although the incidental TCE is small and requires very large samples to reliably detect, this is a case where a small effect can be an important effect. Some models would be all but falsified by a truly absent TCE, whereas others would not. I have edited the discussions of Experiments 1--3 to more carefully describe the null effects as ``reduced to levels statistically indistinguishable from zero'' rather than ``eliminated'' and to prepare the reader for the power issues raised in Experiment 4.

In sum, I think the main contribution of the paper is to provide a serious challenge to the prevailing theories of the contiguity effect. On one hand, the data show that the TCE is not purely a matter of ad hoc strategy. On the other hand, the TCE is far less robust to variations in encoding conditions than many models suggest it should be. 

Below I have complied a list of the other changes made in response to each point raised in the action letter and the reviews. I want to stress my gratitude to yourself and the reviewers for your very helpful feedback. Given that this feedback to theoretically substantive changes to the paper, I hope it is appropriate that I have included a thank you to yourself, Dr. Nairne and Reviewer 1 in the acknowledgments. I hope you agree that the resulting changes have strengthen the manuscript considerably.

\vspace{20pt}

\textbf{\large{Action Letter}}


\begin{enumerate}

\item
	% QUOTE: Reviewer 1 comments about them level of detail of the two theoretical accounts, but I'll go further: the introduction is too short. Not everyone is familiar with the TCE and it needs to be described and put into context. Similarly, the two possible explanations are only vaguely described; they need to be described more completely so that the predictions are better understood.
	You echoed Reviewer 1's request for a more detailed introduction. As requested, the new introduction includes a fuller description of the TCE (some of this was moved from the methods to the introduction) and its context within the memory literature (Page \pageref{TODO-1}), % just include your standard speal like from the 2014 papers
	% QUOTE:Similarly, the two possible explanations are only vaguely described; they need to be described more completely so that the predictions are better understood.
	an extended discussion of the two theories and their competing predictions (Page \pageref{TODO-2}),
	% QUOTE: Also missing is a statement separating the encoding conditions (incidental vs. intentional) and the retrieval conditions (direct vs. indirect test). You mention incidental, but the reader has to infer that the critical test conditions will be (direct).
	a clarification that the study involves incidental encoding followed by a \emph{direct} test of recall (Page  \pageref{TODO-3}),
	% QUOTE: you don't describe the Nairne et al. experiment and the reader is therefore unable to know how your study compares to their study.
	and a more detailed description of the Nairne et al. (2017) experiment (Page \pageref{TODO-4}).

\item
	%QUOTE: Reviewer 1 also noted the very small contiguity effect in the explicit conditions relative to other studies. This also needs to be addressed. Is it because you have only 1 list? What if you run a model such that only one list is simulated: does that give smaller effects? Put another way, is it theoretically possible to get a large (> 1) TCE with only 1 list?
	You also highlighted Reviewer 1's observation that the contiguity effect reported here is much smaller than the effect reported in previous studies, even in the explicit encoding conditions. I now discuss this on page \pageref{TODO-5}. In brief, the effect size very likely reflects the fact that it is a single list study. In an analysis of practice effects in another data set (Healey \& Kahana, submitted), I have found that on their first list of a large in-lab explicit encoding study, participants show a level of contiguity every similar to that found in the Explicit condition here, and that by the $12^{th}$ list it increases to the level typically reported. 

	In the revision, I have added an analysis of participants' second list (see the new Figure on Page \pageref{TODO-6}) which shows numerically larger TCEs in both conditions though the increase is not statistically significant. These list 2 analyses are reported in the main text for Experiment 1 and, to keep the paper relatively streamlined, in the supplemental materials for Experiments 2 and 3.

	As I note in the manuscript, the fact that the magnitude of the TCE changes with task experience is yet another challenge to most existing models, because they have no mechanisms that depend on task experience---In fact many models reset from list-to-list and therefore would predict identical contiguity regardless of list number.

	To ease comparison with previously reported temporal contiguity effects, I have changed the aspect ratio of the lag-CRP figures to have equal height and width, as is typical in most previous publications. This has the effect of accentuating the slope. 

	% (as you asked)
	% % Perhaps make put a version of this in the paper
	% For most models, (e.g., TCM) contiguity would be idential regardless of the number of lists simulated because the effectively erase their memory after each list (as a simplyfing assumption and a practical). Even more recent models (e.g. CMR2) which do simulate multiple lists, would produce roughly equal contiguity from trial-to-trial unless the allowed the parameters which control contiguity to change with task experience.

\item
	% QUOTE: Like Reviewer 1 (see also a comment by Reviewer 2), I wondered if you observed standard-looking serial position functions. These have been reported previously for a single list learned undering incidental instructions (e.g., Neath, 1993, Exp. 1). This may be important if, for example, you get two different shaped functions under explicit and implicit, and get a TCE in the former but not in the latter. Similarly, an output order analyses may be informative. Geoff Ward and colleagues have a number of papers on this. This may also be relevant to the size of the TCE. These sorts of additional analyses may also shed light on why the results of Experiment 3 differs from those of Experiments 1 and 2.
	As requested by yourself and all three Reviewers, the revision includes new figures showing serial position functions for each condition. I have also included probability of first recall functions in these new figures. The data, which are discussed on page~\pageref{SPCtalk}, show that whereas participants in the explicit recall condition tend to initiate recall from the beginning of the list (as is typically observed in delayed free recall), participants in the incidental conditions tend to initiate recall from on of the last few items presented. The SPCS show that the explicit conditions also tend to show higher recall for primacy items. This pattern is similar to previously reported findings with incidental encoding \cite<e.g.,>{MarsWerd72,Neat93,GlenEtal80} and may reelect the influence of rehearsal. The one exception to this pattern is the ``Relational'' incidental encoding condition (where participants attempted to integrate each new item into an ongoing movie). This condition produced a primacy heavy PFR and a more bow shaped SPC than any other condition in the study. In the manuscript, I suggest that differences in rehearsal may contribute to the group differences in the TCE. Although these new analyses do not provide any clear insights in to when the TCE will be observed, I think they add a useful dimension to the paper by tying it more closely to previous work on incidental encoding.


\item
	% QUOTE: Reviewer 2 notes that there are a number of studies that evidence for encoding temporal information under incidental learning conditions. Given this, you may wish to rephrase slightly your questions.
	On page \pageref{TODO-8} I discuss several studies that have found evidence that temporal informating is encoded during memory search by asking participants to reconstrust serial order. I rephrase the question of the current manuscript as whehter or not subject sponteniously use this infomration to guide memory search on a free recall task.

\item
	% QUOTE: While it is laudable to have such a large sample, it is also important to know what the sample is like. What was the mean age? Were they native speakers of English (you are asking them to process and recall English stimuli)? How many were male? How many female? If you did not collect information about this, you need to state that you didn't, and it weakens the results.
	You noted that it is important to know something about the characteristics of the sample, such as age and native language, even when the sample size is high. This information is reported for the new Experiment 4 (Page \pageref{newexp}). The sample was about 58\% female, 97\% native English speakers, 99.5\% completed at least high school, 34\% had completed an associated degree or higher, and had a mean age of 37.

	As I now note on page \pageref{TODO-10}, I did not collect demographic information for the original 3 studies because it was not possible to do so while ensuring participants would finish in under 10 min (increasing beyond 10min would have doubled the cost of the study). For the new experiment, I dropped the second list from the design to make time for a demographic questionnaire. %I also note that although the demographics of the mturk community have been described elsewhere (REFS), lacking information on the particular sample used in Exps 1--3 is a limitation.

\item
	% QUOTE: I have a quibble about a statement on page 11: "the Explicit condition showed a distinctly peaked lag-CRP" I'm not sure I agree with this: For the negative lags in particular, the line looks flat, not peaked. Please provide some statistical test to support this claim. 
	On page \pageref{done-11} I have edited the sentence ``...the Explicit condition showed a distinctly peaked lag-CRP z(TCE) significantly above zero...'' to drop drop the claim about peakedness because, as you point out, it was not supported by a statistical test. Now it simply says ``...the Explicit condition had a lag-CRP z(TCE) significantly above zero...''
	% QUOTE: Similarly, I have another quibble with how you treat the correlation between overall level and magnitude of the TEC. In Figure 4, you report that there is no statistical evidence for a correlation, p = 0.586. Yet in the text you talk as if the correlation exists and is significant ("Although the correlation is positive, it is quite small ..."). Again, please either report additional statistical analyses such that you justify talking about a non-significant correlation, or change the text to match the outcome of the statistical test you did report.
	Similarly, on page \pageref{done-12} when discussing the across-condition correlation between recall accuracy and the TCE, instead of saying ``Although the correlation is positive, it is quite small...'' I simply say ``The correlation is quite small and non-significant...'' %(in fact, with the addition of the three new conditions it is even smaller than in the original submission)

\item
	% QUOTE: At some point, you may wish to mention that incidentally-learned information can be recalled better than intentionally-learned information depending on the processing done at encoding (e.g., Eagle & Leiter, 1964). This has implications for a possible qualification of the statements about overall recall levels.
	On page \pageref{newcite} I \citeA{EaglLeit64} when discussing the influence of type of processing on recall success after incidental encoding.
	% Eagle, M., \& Leiter, E. (1964). Recall and recognition in intentional and incidental learning. Journal of Experimental Psychology, 68, 58-63.

\item
	% QUOTE: p. 10: Please use colours such as white and light grey (rather than black and dark grey) so that the error bars are visible.
	For bar plots with black bars, I have changed the color of the error bars to make them more visible. 


\end{enumerate}


\vspace{20pt}

\textbf{\large{Reviewer 1}}



\begin{enumerate}

\item 
	% QUOTE: First, it would be nice if differences in the two accounts (controlled vs. automatic) of the temporal contiguity effect (TCE) were more fleshed out. I understand the general idea, but in the current paper the differences seem rather vague. What is it about controlled processes that influences the size of the TCE? Is it strategies or intent? The author suggests that it is not just intent, but there is really no indication of how different strategies would influence it. Similar for the automatic processing account. How/why does this account predict the effect? I have read the prior papers, but much more needs to be said in the current study regarding the two accounts and what the overall theoretical advance is other than current theories are lacking.
	The reviewer requested an elaboration of how temporal contiguity is produced under each of the two theoretical accounts of the temporal contiguity effect (controlled/strategic vs automatic) and the differences between the proposed mechanisms. As I describe in more detail above, the revised introduction includes an expanded section on each theory including examples of how contiguity is claimed to arise under each (Page \pageref{TODO-2}). 

\item
	% QUOTE: Second, one interesting finding in all the experiments was how small the temporal contiguity effect was. Is this because only one list was used? The author and other work by Kahana and collegues has typically found much larger contiguity effects where the lag +1 conditional response probability is closer to .30. Similarly, typically contiguity effects are asymmetrical, but most of the current effects look pretty symmetrical. I'm aware that there are some studies that have found symmetrical effects, but it is not clear what is going on in the current data. Again, much more discussion is needed to better explain what is going on with the current data.
	As requested, I have included a discussion of how the size and symmetry of the TCE in the current study compares to that observed in previous work, including a recent report that the TCE does tend to be small on the first list and grow with task experience (Page \pageref{TODO-5}). This discussion includes an analysis of subject's second lists (see my response to the Action Letter for details). 
	
\item
	% QUOTE: Finally, are serial position and probability of first recall curves similar across the different conditions? Examining these other patterns of performance may give some insight into the current results.
	I have added serial position and probability first recall curves for all of the conditions (again described in more detail in my response to the action letter). These suggest that incidental encoding encouraged participants to focus their initial memory search on recency items at the expense of primacy items (Page~\pageref{SPCtalk}).



\end{enumerate}


\vspace{20pt}

\textbf{\large{Reviewer 2}}

\begin{enumerate}

	\item 
	% QUOTE: Although not reviewed here, there is already substantial evidence that people encode and retain temporal position and order information under incidental learning conditions (e.g., Nairne, J. S. (1991). Positional uncertainty in long-term memory. Memory & Cognition, 19, 332-340); in this work, temporal contiguity-like effects can be seen in the errors gradients that are produced when people are asked to reconstruct the original orders of presentation. So, the questions are really (1) under what conditions do people use this information strategically to drive recall? and (2) can differential temporal coding, or strategic use of that encoding, be used to explain large and consistent condition differences in recall (such as levels of processing or, in the case of Nairne et al. (2017), the difference between survival processing and controls). The answer to the second question is clearly no, as shown here and in Nairne et al., and the current experiments provide no answer to the first question.
	The reviewer pointed to work showing that when incidental encoding is followed by a direct test of temporal order memory, accuracy is fairly high, suggesting that temporal information survives incidental encoding. On page \pageref{TODO-8} of the revision I discuss these findings along with the more recent demonstration of this from the order reconstruction task in experiment 3 of Nairne et al. (2017). I also note, however, that this interpretation of the data is not without its critics (Hintzman 2015), which suggests that there is value in providing converging evidence. Thus, I see the contributions of the current manuscript as 1) providing converging evidence for the existence of incidentally encoded temporal associations and 2) showing that subjects spontaneously use these associations to guide memory search even when not prompted to do so with a direct order test. 

	Moreover, I completely agree with the reviewer that a key question is now "under what conditions do people use this information strategically to drive recall?" I think a whole research program could be devoted to this question, and the new Experiment 4 takes a first step in this direction. In this experiment I took the Reviewer's suggestion of holding encoding conditions constant but varying retrieval instructions (either free recall or serial recall). Both sets of encoding instructions produced a significant TCE and there was no significant difference between conditions in the size of the effect (though there was a trend, it was not reliable with 500+ subjects per condition). These data suggest that participants may be using something close to all the available temporal information during free recall. Clearly, this experiment opens as many questions as it closes (e.g., does output order interference prevent subjects from fully utilizing temporal associations in serial recall; can the TCE be boosted by encouraging a subjects to think about temporal associations during memory search without enforcing strict serial recall, how do these retrieval factors interact with encoding conditions, etc.), and I hope one contribution of this manuscript is to motivate the field to work on these questions.


	\item
	% QUOTE: I don't know why the task instructions are placed in supplemental materials. They should be in the methods sections.
	The reviewer asked why the task instructions were placed in supplemental materials and suggested they be moved to the main text. I placed them in supplemental materials because they are quite long (over 10 pages) and repetitive (much of the same language is repeated verbatim from condition to condition). I worry that including them in the main text will make the paper too long and reduce its readership. I am, of course, willing to move them if the reviewer and editor feel strongly.

	\item
	% QUOTE: The discussion of how to calculate the TCE is confusing and will be opaque to most readers. A specific example would help.
	I have added a specific example illustrating the the calculation of the z(TCE) measure on page \pageref{TCEex}. In that section I have also clarified why this procedure is necessary: under special cases, the lag-CRP can appear peaked when there is in fact no TCE and in other cases it can appear flat when there is in fact a TCE.

	\item
	% QUOTE: Some readers will probably want conventional statistical analyses which are absent here.
	The reviewer suggested that some readers might prefer conventional statistical analysis instead of the confidence intervals I used throughout the manuscript. I have added t-test and ANOVAs in cases where significance cannot be quickly assessed by examining the confidence intervals (e.g., Page \pageref{t1})

	% And the Experiment 4 includes an ANOVA.

	% I have, however, opted to maintain the reliance on confidence intervals rather than t-tests as I find it makes the results section more readable without loosing any information or rigor. 



	

\end{enumerate}





\vspace{20pt}

Thank you for your time and consideration.

\vspace{10pt}

Sincerely,

\vspace{10pt}

Karl Healey\\
khealey@msu.edu\\
Michigan State University\\
Department of Psychology

\bibliography{healey_lab}

\end{document}
