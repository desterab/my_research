\documentclass[12pt]{article}
\usepackage{color,soul,xcolor}
\usepackage{geometry}
\usepackage{graphicx}
\usepackage{apacite}
\usepackage{csquotes}

% to use page refs from the manuscript tex file
\usepackage{xr}
\externaldocument{Heal16implicit}


% Miscellaneous dimensions&
\geometry{letterpaper,left=.75in,right=.75in,top=.75in,bottom=.75in,centering}
\setlength{\parskip}{1ex}
\setlength{\parindent}{0em}
\setlength{\headheight}{15pt}

\bibliographystyle{apacite}

\begin{document}


\textbf{Scenario Processing Task Instructions:}

\begin{displayquote}


We are trying to find a set of words that are neither too easy nor too hard for people to process.
            We want to use these words in an upcoming study of brain activity during word processing.
            Your responses today will help us select appropriate words.

        You will be asked to make a judgment about each word in the list.
            Specifically, you will decide how relevant each word is to an imaginary scenario.

        We would like you to imagine that you are planning to move to a new home in a foreign land.
            Over the next few months, you’ll need to purchase a new house and find help transporting your belongings.

        For each word you see, think of how relevant the word would be for you in this moving situation.
        Some of the words may be relevant and others may not---it’s up to you to decide.


        You will see each word for only a few seconds.
            Therefore, for some words, you will find it easy to judge its relevance.
            For other words, you will find it difficult. For each word, you will try to judge its relevance
            and decide if it is easy or difficult.

        There are two possible choices for each word: "Yes" and "No".
            You will respond by pressing the "y" key for "Yes" or the "n" key for "No".

        For example, if we asked you to judge the word "owner",
            you would press "y" if you can easily judge its relevance, and "n" if you find it difficult.

        The task will automatically advance to the next word after a fixed amount of time.
            There will be a brief screen showing a cross, "+", between each word presentation.

        To ensure participants are thinking carefully about each word, we will
          randomly pause the task and ask you to type a description of how you made your judgment. [\emph{we never actually paused the experiment}]

\end{displayquote}





\textbf{Size Processing Task Instructions:}

\begin{displayquote}


         We are trying to find a set of words that are neither too easy nor too hard for people to process. We want to use these words in an upcoming study of brain activity during word processing. Your responses today will help us select appropriate words.

       You will be asked to make a judgment about each word in the list.
       Specifically, you will decide whether or not the word refers to an object that could
       fit into a regular shoebox.

       You will see each word for only a few seconds. Therefore, for some words, you will find it easy to decide if it fits in a shoebox. For other words, you will find it difficult. For each word, you will try to decide if it fits in a shoebox and indicate whether it was easy or difficult. 

      There are two possible choices for each word: "Yes" and "No". You will respond by pressing
      the "y" key for "Yes" or the "n" key for "No".

      For example, if we asked you to judge the word "owner", you would press "y" if you can easily decide if it fits in a shoebox, and "n" if you find it difficult.

      The task will automatically advance to the next word after a fixed amount of time.
      There will be a brief screen showing a cross "+" between each word presentation.

      To ensure participants are thinking carefully about each word, we will randomly pause the task and ask you to type a description of how you made your judgment. [\emph{we never actually paused the experiment}]

\end{displayquote}






\textbf{Deep Processing Task Instructions:}

\begin{displayquote}

We are trying to find a set of words that are neither too easy nor too hard for people to process. We want to use these words in an upcoming study of brain activity during word processing. Your responses today will help us select appropriate words.

You will be asked to make a judgment about each word in the list. Specifically, you will decide whether or not you can easily generate a ``mental movie'' related to that word.

When you read a word, it can trigger many different thoughts. For example, if you see the word ``BASEBALL'' you might have a mental image of a baseball, you might hear the crack of a bat hitting a ball, you might think of related concepts like ballpark, players, and fans. You might have a mental image of a baseball game that includes the sounds, sights, and smells of a baseball stadium. You might think back to personal experiences related to baseball such as the last time you watched a game, or learning to play baseball as a child. All of these thoughts may be associated with positive or negative emotions.

For each word you see, allow it to activate as many different thoughts as possible. Then use these thoughts to generate a mental movie (like a detailed image of spending an afternoon at a baseball game or what it is like to be a player on a baseball field). 

You will see each word for only a few seconds. Therefore, for some words, you will find it easy to generate a mental movie. For other words, you will find it difficult. For each word, you will try to form a mental movie and decide if it is easy or difficult. 

There are two possible choices for each word: "Yes" and "No". You will respond by pressing the "y" key for "Yes" or the "n" key for "No".

For example, if we asked you to judge the word "owner", you would press "y" if you can easily generate a scene, and "n" if you find it difficult.

The task will automatically advance to the next word after a fixed amount of time. There will be a brief screen showing a cross, "+", between each word presentation.

To ensure participants are thinking carefully about each word, we will randomly pause the task and ask you to type a description of your mental movie. [\emph{we never actually paused the experiment}]

\end{displayquote}
\end{document}