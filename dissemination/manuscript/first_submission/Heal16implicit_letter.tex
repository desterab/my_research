\documentclass[12pt]{article}
\usepackage{color,soul,xcolor}
\usepackage{geometry}
\usepackage{graphicx}
\usepackage{apacite}

% to use page refs from the manuscript tex file
\usepackage{xr}
\externaldocument{Heal16implicit}


% Miscellaneous dimensions&
\geometry{letterpaper,left=.75in,right=.75in,top=.75in,bottom=.75in,centering}
\setlength{\parskip}{1ex}
\setlength{\parindent}{0em}
\setlength{\headheight}{15pt}

\bibliographystyle{apacite}

\begin{document}

%%%%%%%%%%%%%%%%%%%%%%%%%%%%%%%%%%%%%%%%%%%%%%%%

%INITIAL SUB EXAMPLE:
Dear Dr. Lindsay,
 
%Example of how to cite a change on page \pageref{change1} of the manuscript file.

I ask that the attached manuscript, ``Temporal Contiguity in Incidentally Encoded Memories'', be considered for publication as a Research Report in Psychological Science. In the manuscript, I present the results of three experiments that test competing theories of the Temporal Contiguity Effect. The Temporal Contiguity Effect is the finding that recalling one memory tends to trigger retrieval of other memories that were encoded nearby in time. Some theories suggest that this effect is an artifact of laboratory tasks and should disappear under incidental encoding. Other theories, however, suggest that the effect tells us something fundamental about memory and should be observed under many encoding conditions. The results in the attached manuscript show that the effect can indeed be observed under incidental encoding, suggesting that the first class of theories are wrong. But they also point to some important limitations in the second class of theories. Thus, the results make two important theoretical contributions.

 All data analyzed in the report is freely available to yourself and the reviewers at https://cbcc.psy.msu.edu/data/Heal16implicit.csv, and will be made publically available if the manuscript is accepted for publication.

 I request that Dogulas Hintzman not serve as a reviewer. Given that the manuscript is directly related to work he has published, I feel obliged to provide a detailed explanation. We have a history of strong pre-theoretical disagreements that extend well beyond the issues raised in the manuscript. During the review of other manuscripts, these disagreements have proven to be contentious, inflexible, and irreconcilable. In my view, this history makes it difficult for us to provide unbiased reviews of each other's work. In the online system, I have suggested several reviewers who are sympathetic with Hintzman's theoretical perspective (e.g., Nash Unsworth) who I am confident can provide a critical, but unbiased, review.

\vspace{10pt}

%I suggest the following scientists as reviewers due to their expertise in human memory:
%
%Gene Brewer, Arizona State University, gene.brewer@asu.edu
%
%Gordon Brown, The University of Warwick, G.D.A.Brown@warwick.ac.uk
%
%Simon Dennis, University of Newcastle, simon.dennis@newcastle.edu.au
%
%Simon Farrell, University of Western Australia, simon.farrell@uwa.edu.au
%
%Mark McDaniel, Washington University in St. Louis, mmcdanie@artsci.wustl.edu
%
%Ian Neath, Memorial University, ineath@mun.ca
%
%Nash Unsworth, University of Oregon, nashu@uoregon.edu
%Geoff Ward, University of Essex, gdward@essex.ac.uk
%
%James nairne
%
%gutches
%
%thapir

\vspace{20pt}

Thank you for your time and consideration.

\vspace{10pt}

Sincerely,

\vspace{10pt}

Karl Healey\\
khealey@msu.edu\\
Michigan State University\\
Department of Psychology

\bibliography{healey_lab}

\end{document}
%
%% REPLY EXAMPLE:
%
%Dear Dr. Unsworth,
%
%We thank you and the reviewers for your comments on our manuscript  ``Contiguity in Episodic Memory'' (XLM-2016-2812). In the manuscript we argue that the temporal contiguity effect is an important benchmark finding that theories of memory should explain, review the existing literature, present several novel analyses of archival datasets, present results from two new experiments, and discuss how existing theories can handle this broad pattern of findings.
%
%As noted in your decision letter, the reviews ranged from highly positive (e.g., Reviewer 1 raised no major issues) to strongly critical. The diversity and strength of opinions expressed in the reviews made it clear that we had overstated our case in the original manuscript.
%
%Therefore, in response to the issues raised by yourself and the reviewers, we extensively revised the manuscript.  Chief among these is adding a new experiment that replicates and extends the, as yet unpublished, finding that contiguity is absent under implicit encoding conditions. We discuss the implications of this important finding for the various theories. In brief, we acknowledge, as Reviewer 2 suggests, that finding points to a limitation of temporal context models (and other formal episodic memory models).
%
%As requested we have complied a list of the changes made in response to each point raised in the action letter and the reviews. In the manuscript, we have used
% \setstcolor{red}
%  \color{red}red font\color{black}~to highlighted all additions and \st{strikethrough} to highlight all deletions.


% \vspace{20pt}

% \textbf{\large{Action Letter}}

% We begin by outlining changes we have made in response to several points you raised in addition to the issues raised by the reviewers.

% % QUOTE: For example, much of the current review is based on analyses from the PEERs project. To what extent have these analyses in some form been reported in prior papers? It is critically important that it is made clear what analyses are new (or reanalyses) and what analyses are old.
% \textbf{1.} You asked that we clarify which analyses are completely novel, which are replications of previous findings with new data, and which are reported in existing papers. To make it as easy as possible for the reader to digest this information we have created a new table that lists each analysis, whether the finding is novel, a replication with new data, or a representation of previous work. %TODO: MAKE THIS TABLE


% \textbf{2.} You pointed out that whereas for the PEERS data reported in the manuscript there is a robust contiguity effect regardless of semantic relatedness (Figures 2N and 2O), work by Siddiqui and Unsworth (2011) had found more modest temporal contiguity effects effects when lists were composed of positive, negative, or neutral words. We include a discussion of this finding on page \pageref{AL.2}: "Paste text here!" 

% % QUOTE: Furthermore, it is repeatedly noted that IQ correlates with the temporal factor score. But, to what extent is this relation somewhat spurious given that the IQ is related to recall and recall is related to the temporal factor score? That is, recall ability is the shared source of variance to both. There may be nothing special about the relation between the temporal factor score and IQ scores.
% \textbf{3.} You pointed out that the correlation between temporal factor score and IQ could arise from the fact that overall recall is correlated with both temporal factor and IQ. This is in fact the case, as we have previously shown in Figure 4 of \citeA{HealEtal14}, overall recall scores mediate the correlation between temporal contiguity and IQ. But we maintain that there is something special about temporal contiguity because none of the other measures of recall dynamics we considered (semantic contiguity and measures of how participants recall initiation) correlated with IQ once temporal contiguity is accounted for. In our view, it is an open question whether temporal contiguity per se is special or whether there is some underlying ability (e.g., the ability to constrain search sets, as you have argued), that is critical. In other words, we think the evidence justifies saying temporal contiguity is special, even if it does not justify saying temporal contiguity is casual. On page \pageref{AL.3} we clarify: ``It is important to note that these correlations do not imply causality (it could be that both temporal contiguity and IQ share variance with some third factor) but, critically, semantic clustering does not correlate with IQ. Although the direction of the relationship remains unclear, there is something special about temporal contiguity in predicting recall and IQ.'' %TODO: refine this!

% \vspace{20pt}

% \textbf{\large{Reviewer 1}}

% Reviewer 1 was highly positive and highlighted the fact that the manuscript is ``is timely given the recent Hintzman (2015) article'' and suggested it had ``the potential to provide a highly suitable summary reference from which alternative explanations and viewpoints can depart.''. 

% \vspace{20pt}

% \textbf{\large{Reviewer 2}}

% \textbf{1.} Reviewer 2 indicated that the paper needed to more clearly lay out the key questions it is intended to address. As the reviewer suggested, the manuscript takes aim at several issues that have arisen in the recent literature. On page \pageref{R2.1} we have added a paragraph that clearly specifies these: "FILL IN". and in the discussion we return to these on page \pageref{R2.1b}: "FILL IN" %TODO: Decide what to say here

% \textbf{2.} The Reviewer indicated that our description of Retrieved Context Models did not give the reader an accurate sense of the model's complexity because we omitted descriptions of some parameters. The only major component of the model we did not describe was how it uses the match between each item in its memory and the current state of context to make a decision about which item to recall. In a related point, the reviewer suggests that the model fails to simulate retrieval monitoring processes that ensure participants do not repeatedly recall the same items. While it is true that some models simply ignore repetitions, it is not true of the most recent implementation of retrieved context models, which explicitly model a retrieval monitoring and editing process. We have addressed both the originally un-described retrieval process and the issue of repetitions on page \pageref{R2.2}: ``In addition to context drift, association formation, and context reinstatement, the models also simulate the competitive retrieval process via a evidence accumulation process. More recent versions of the model include mechanisms to screen for intrusions by comparing the context associated with each candidate recall with the current state of context and suppress repetitions by dynamically sets the retrieval threshold of each item as the recall period progresses to allow items that were recalled earlier in the period to participate in, but not dominate, current retrieval competitions \cite{HealKaha15,LohnEtal14}.''

% \textbf{3.} The Reviewer requested that the correction for serial position effects be conducted for all analyses. We have conducted these analyses and added the following description of the results on page \pageref{R2.3}: "FILL IN". %TODO: rerun all analyses including temporal factor correlations
%  Because the correction does not change the key findings, to facilitate comparison between the current work and prior analyses of the contiguity effect, we show the non-corrected results in the main figures and include versions of the figure with the correction as supplemental material.

%  \textbf{4.} The Reviewer argues that to test whether contiguity effects are ``attributable to an automatic process'' or ``to study strategies that subjects use because they know how they are going to be tested.'' we must run an experiment ``in which subjects do not know or suspect that their memory for the words will be tested''. As the reviewer points out, Nairne et al. have run such an experiment, though the results are not yet published. Therefore, in addition to discussing the Nairne result, we have run a new study replicating the finding. A full description of the methods is on page \pageref{R2.4}, but briefly: All participants completed two free recall lists, in the explicit condition they were given standard instructions to study the words and in the implicit condition all mention of memory was omitted. Replicating Nairne et al., we found a clear contiguity effect in the explicit condition, but not the implicit condition. Both groups, showed clear contiguity on a second trial. These results are presented in Figure X and described on \pageref{R2.4b}. On page \pageref{R2.4b} we discuss the theoretical implications of this result: ``FILL IN''. %TODO: what do we want to say about this result?!?

%  %NEXT: finish #5 and #6 from hintzmans review
%  \textbf{5.} The original manuscript included a paragraph that expressed our opinion that data from certain paradigms (e.g., spacing judgments and position judgments) are difficult to interpret as evidence either for or against temporal contiguity because contiguity-generating episodic memory models, like retrieved context models, have not been extended to these paradigms. The Reviewer felt that this position was unwarranted. We disagree and have added the following clarification on Page \pageref{R2.5}: ``Under the models of episodic memory we have discussed, contiguity emerges as a consequence of attempting to recall episodes. These other paradigms add the additional requirement to make judgments about episodes. These judgments are beyond the ken of the existing models. And without clearly specifying the processes that translate into judgments there is simply no principled way to make strong predictions about whether, or how much, contiguity one would expect to see in the data.''

% \textbf{6.} The reviewer requested that we add a discussion of data from two studies by Glenberg \& Bradley. On page \pageref{R2.6} we have added: ``FILL IN''%TODO:read the studies and write someting

% \textbf{7.} Finally, the reviewer indicated that our claim, near the end of the manuscript, that encoding tasks and intent to learn might modulate contiguity effect seemed inconsistent with the earlier sections that argued contiguity was a inherent property of encoding and search processes. The new study showing that contiguity can indeed be eliminated under implicit encoding conditions obviates this concern, in our view. We now acknowledge, early in the manuscript (\pageref{R2.7}, that processes under conscious control, like encoding processes, can reduce and even eliminate contiguity.

% \vspace{20pt}

% \textbf{\large{Reviewer 2}}
% The Reviewer felt that the manuscript was ``a thoughtful response to Hintzman's recent paper in which he questions whether episodic memory is temporally organized'' and saw ``no reason why the authors should not make their case in publication''. The Reviewer did raise some important concerns, which we have attempted to address in the revision.

% \textbf{1.} The reviewers central concern was with our attempt ``to classify different models in order to derive predictions from classes of models instead of considering each model individually''. We agree that there are important differences between models that we have placed in the same class, as well as important similarities among models we have placed in the same class. We have included a new section that acknowledges that the category boundaries are fuzzy on Page \pageref{R3.1}: ``FILL IN.. include something like: This rough categorization has proven to provide a useful framework in the past \cite{FHM,ReviewCh}. Of course, it is intended as a rough heuristic---one that is necessitated by the fact that none of the models have been applied to the range of paradigms. In our view, an important goal for the field as a whole, and proponents of specific models, and the true test lies in extending specific models to simulate the paradigms in question.'' %TODO: write someting that promenently features a nice comment about the malmberg model. Something like this quote from the review: "The Malmberg and Lehman model predicts that the control processes used to encode items plays a crucial role in the degree to which to events are associated. Associative encoding reflects the goals or strategies adopted by the subject, and it is clearly assumed that under some conditions that associative encoding of temporally adjacent events will be actively avoided via compartmentalization"

% \textbf{2.} The reviewer pointed out that there was no reference for the ``Law of Similarity'' quote in the first paragraph. Although we put our statement of the las in quotation, it was actually original text and not quoted from any source. We have removed the quotes.

% \textbf{3.} The Reviewer wondered if Hume's reference to associations was in reference to learning (e.g., conditioning or semantic memory) rather than to episodic memory. Specifically, that ``Our inductive expectation about what the future holds is usually available without reference to any particular event.'' We agree that people make inductions based on many events, not usually single events. But Hume does not make a clear distinction between episodic and non-episodic memory and our interpretation is that he is referring to the abstraction of general rules from the accumulated memory of many individual events. 

% %QUOTE: Since deblurring has never been explained in great detail, with the exception of Goebel and Lewandowsky's (1991) attempt, it is really difficult to make any concrete statements about what effect deblurring would have on retrieval in general and contiguity specifically.
% \textbf{4.}  We suggested that ``in a model in which retrieval is accomplished by correlating a probe with a memory vector (e.g. Lewandowsky \& Murdock, 1989), probing with the recalled item would produce a blurred representation that, on average, is most similar to the item at lag 1 but also somewhat similar to items at remote lags.''. The reviewer argued that the deblurring processes has not been explained in detail (except Goebel and Lewandowsky 1991), it is difficult to make concrete statements. We agree that precise predictions are difficult to make, but it seems quite likely that the process would provide some support for items at remote lags providing a plausible avenue for the models to produce contiguity. %TODO:??? Ask Mike what he thinks

% \textbf{5.} The reviewer pointed out that our description of dual store models omitted some details that have been added to the most recent version, which can modulate the contiguity effect. On page \pageref{R3.5} we have added: ``Current dual-store models have specified some of the control processes that govern how items are processed. For example, Lehman \& Malmberg's (2011) model includes a compartmentalization process that can focus the buffer on particular time periods and can thereby modulate contiguity.'' %TODO: read their description again. 

% \textbf{6.} We have added a reference to Lee and Estes (1977) perturbation models to the position coding section \pageref{R3.6}.

% \textbf{7.} We have removed the sentence "All the models we have considered so far assume that the degree of association between two events is a smoothly decreasing function of the amount of time that separated them." from the hierarchical clustering model section because, as the reviewer points out, some models can use control processes to modulate how items are associated. %TODO: remove the sentence!

% \textbf{8.} We have revised the section on hierarchical clustering models to clarify that ``chunks need not be hierarchically structured under some descriptive label (Malmberg \& Lehman, 2013; Murdock, 1982, 1995).'' %TODO: Write someting about this

% \textbf{9.} In the section on retrieved context models on page \pageref{R3.9} we now note the overlap with some models we have classified as dual store: ``Many of the key features of retrieved context models are shared with other models, especially modern dual-store models, which include similar contextual evolution and reinstatement processes for long-term storage and retrieval \cite{Malmberg,Davelar}''

% \textbf{10.} Early in the original manuscript we argued that one should evaluate a model's based on it ability to account for a wide range of findings. The Reviewer argued that accounting for many data points does not automatically count in a model's favor. Rather one must also consider which data points a model is not able to account for. The reviewer specifically drew attention to the finding that contiguity is reduced when pairs of items are studied (Malmerg \& Lehman, 2013). We have revised this section on page \pageref{R3.10} to make this point: ``FILL IN'' %TODO write someting and include mention of MALMBERG and LEHMAN either here or later and quote both in the letter

% Later in original manuscript we included a section on how to further test the retrieved context framework. We have revised that section to more explicitly point out several predictions of the framework that could be falsified by experiments. This include HIGH MEMORY WIHT NO CONTIGUITY (see page \pageref{R3.10b}).

% \textbf{11.} The Reviewer pointed out that some passages gave the impression that we think contiguity should be all or none under a strategic encoding account. On page \pageref{R3.11} we clarify that ``Of course, different strategies may produce different degrees of contiguity (e.g., a strict serieation strategy that one would apply to memorize a phone number would produce more contiguity than would a strategy based on using the items to tell a story). Moreover, even for a particular strategy, it might generate more or less contiguity depending on how effectively the participant implements it and how well-suited it is to the task. Critically for our purposes, any contiguity effect due to such strategies should be eliminated in situations in which participants are unlikely to employ the strategy. That is, strategy generated contiguity should be much less ubiquitous than contiguity generated by a process integral to memory encoding or search.'' 

% \textbf{12.} In the original manuscript we has suggested ``A strategy-based account has no parameters that map naturally on to age-related change...''. The Reviewer felt that this was overstating the case and pointed to several parameters that might be modulated by age. We agree and have removed this sentence from Page \pageref{R3.12}. %TODO: remove this text from paper

% \textbf{13.} On page \pageref{R3.13} we have revised our discussion of contiguity in continual distractor free recall to clarify that some dual-store models can account for the effect: ``Although a basic dual-store model has difficulty with contiguity in continual distractor free recall because the distractors should empty the buffer and prevent formation of temporally graded associations, modern dual-store models can produce contiguity in the presence of distractors by incorporating contextual drift WHAT DOES MALMBEG MODEL DO \cite{Davelar,Malmberg}'' %TODO: finish this off

% \textbf{14.} The Reviewer argued that is was difficult to put too much weight on the Howard et al. finding of contiguity under extremely fast presentation rates because the data is unavailable. WOULD MARC LET US POST THE DATA??!?? The Reviewer also argued strategic encoding might actually be \emph{more} important under ``difficult'' encoding conditions. We agree that some sources of difficult might force participants to rely more heavily on strategy. But as we argue in the revision on page \pageref{R3.14}, extremely fast presentation rates seem unlikely to do so: ``That is, contiguity was seen when words were presented as rapidly as one every 250 ms. Given evidence that perceptual processing of items takes about 180 ms \cite{REF}, is seems quite implausible to us that participants using the few remaining milliseconds to engage in the sorts of complex strategies that have been suggested to create contiguity (e.g., rehearsal, telling a story).''

% %QUOTE:P.28. On the topic of semantic versus temporal competition, it appears that the asymmetry of the contiguity is disrupted when the semantic associate appears at a longer lag. This is predicted by dual-store models if the prior item was not rehearsed with the subsequent item.
% \textbf{15.} The Reviewer note that in Figure 2 panel O the asymmetry is reduced when an associate is available at a remote lag. %TODO: What to say??!?!?

% \textbf{16.} Regarding our discussion of contiguity in recognition, the Reviewer argued that we have no direct evidence that items were ``recalled''. Unfortunately, there was a typo in the manuscript and we used ``recall'' when me meant ``recognition''. The argument that temporal associations formed between items during study should facilitate recognition (not recall) when list neighbors are probed in succession. This argument holds whether or not participants successfully recollected the first probed item---all that is required is that they successfully recognized it. We have corrected this unfortunate typo on page \pageref{R3.16}.
% %TODO: in ``They found that transitions from recalled items for which the subject "remembered"'' and ``aving two successive probes be from adjacent list positions should help recall''  change RECALLED to endorsed. READ THE WHOLE SECTION CAREFULLY!!

% NEXT: read through, make edits and start making the changes to the text!!


% \vspace{20pt}

% Thank you for your time and consideration.

% \vspace{10pt}

% Sincerely,

% \vspace{10pt}

% Karl Healey\\
% khealey@msu.edu\\
% Michigan State University\\
% Department of Psychology

% \vspace{10pt}

% Michael J. Kahana\\
% kahana@psych.upenn.edu\\
% University of Pennsylvania\\
% Department of Psychology 
 
