\documentclass[man,natbib,floatsintext]{apa6} %apa 6th edition format

% watermark for first page
% \usepackage[firstpage]{draftwatermark}
% \SetWatermarkText{In Prep}
% \SetWatermarkScale{.75}
% \SetWatermarkColor[gray]{0.88}

\usepackage{amstext,amssymb,graphicx,bm,soul,color,url,lscape,rotating,setspace,csquotes,pdflscape,rotating}
% \DeclareDelayedFloatFlavor{sidewaystable}{table}
% \DeclareDelayedFloatFlavor{sidewaysfigure}{figure}


\usepackage[space]{grffile}

% as setup by apacite, natbib puts extra spaces between the commas and semicolons in the cites. This fixes it:
\setcitestyle{citesep={;},aysep={,}}
 
% Run texcount on tex-file and write results to a file
\newcommand\wordcount{\input{wordcount.sum}}

% environment to display a page in landscape in the final pdf
\newenvironment{rotatepage}%
    {\pagebreak[4]\global\pdfpageattr\expandafter{\the\pdfpageattr/Rotate 180}}%
    {\pagebreak[4]\global\pdfpageattr\expandafter{\the\pdfpageattr/Rotate 0}}%

% commands for inserting values determined by analyses scripts.
\newcommand\shoeExplicit{290}
\newcommand\shoeIncidental{386}
\newcommand\doorExplicit{137}
\newcommand\doorIncidental{255}
\newcommand\Movie{384}
\newcommand\Relational{409}
\newcommand\Scenario{331}
\newcommand\Animacy{325}
\newcommand\Weight{322}
\newcommand\shoeExplicitAware{--}
\newcommand\shoeIncidentalAware{47}
\newcommand\doorExplicitAware{--}
\newcommand\doorIncidentalAware{43}
\newcommand\MovieAware{55}
\newcommand\RelationalAware{98}
\newcommand\ScenarioAware{32}
\newcommand\AnimacyAware{31}
\newcommand\WeightAware{31}
\newcommand\shoeExplicitIncluded{290}
\newcommand\shoeIncidentalIncluded{339}
\newcommand\doorExplicitIncluded{137}
\newcommand\doorIncidentalIncluded{212}
\newcommand\MovieIncluded{329}
\newcommand\RelationalIncluded{311}
\newcommand\ScenarioIncluded{299}
\newcommand\AnimacyIncluded{294}
\newcommand\WeightIncluded{291}
\newcommand\shoeExplicitPrec{0.46 (0.16)}
\newcommand\shoeIncidentalPrec{0.40 (0.16)}
\newcommand\doorExplicitPrec{0.47 (0.17)}
\newcommand\doorIncidentalPrec{0.47 (0.14)}
\newcommand\MoviePrec{0.38 (0.15)}
\newcommand\RelationalPrec{0.44 (0.18)}
\newcommand\ScenarioPrec{0.38 (0.14)}
\newcommand\AnimacyPrec{0.37 (0.14)}
\newcommand\WeightPrec{0.42 (0.15)}


% counter for panels of crp matrix figure
\newcounter{crppanel}

% commands for making margin notes marked with authors initials
\setlength{\marginparwidth}{30pt}
\newcommand{\mkh}[1]{\marginpar{\scriptsize \textcolor{red}{MKH: #1}}}

\title{Temporal Contiguity in Incidentally Encoded Memories} 

\author{M.\ Karl Healey}

\affiliation{Michigan State University}

\shorttitle{Contiguity with Incidental Encoding}

%\journal{???}

\authornote{M. K. Healey is the sole author of this article and is responsible for its content. I thank Mitchell Uitvlugt, Kimberly Fenn, and SOME OTHER PEOPLE for helpful discussions.
%Correspondence concerning this article should be addressed to M. Karl Healey (khealey@msu.edu) at Michigan State University of Pennsylvania, Department of Psychology, 316 Physics Road, East Lansing, MI.
\begin{flushleft}
phone: 517-432-3107\\
Version of \today\\
\wordcount Words excluding Method and Results (Approximate due to use of \LaTeX)
\end{flushleft}
}


\abstract{Similarity is a key principle of memory: recalling one event triggers recall of similar events. The amount of time separating two events is believed to be a powerful determinant of similarity. If so, there should be a strong Temporal Contiguity Effect whereby thinking of one event triggers recall of other events experienced nearby in time. This effect has been observed, but its source is a matter of debate. Is it due to task-general automatic processes that operate whenever new memories are formed? Or is it due to task-specific encoding strategies that operate only during intentional rote learning? We test these theories by looking for the Temporal Contiguity Effect when there is no intent to memorize, as is often the case outside the laboratory. Three experiments show that it is not the intent to memorize per se, but rather how subjects process information while incidentally learning it that generates temporal contiguity.}

\keywords{episodic memory; free recall; temporal contiguity}

\begin{document}
\maketitle

%How we encode information powerfully impacts our success in remembering it \citep{CraiLock72,TulvPear66}.
Control processes \citep{LehmMalm13,RaaiShif81} allow us to strategically process information during memory encoding, maximizing recall \citep[e.g.,][]{Unsw16}. Little is known about how control processes impact one of the oldest and most influential observations about memory: Recalling one event tends to trigger recall of other events experienced nearby in time \citep{Aris,Bowe72,Kaha96}.

This Temporal Contiguity Effect (TCE) manifests in many tasks \citep{DaviEtal08,SchwEtal05}. For example, in free recall, you study words presented serially and recall them in any order. After recalling the word studied in the $5^{th}$ serial position, your next recall is much more likely to be from the $6^{th}$ or $4^{th}$ position than more distant positions \citep{Kaha96}.

The TCE has shaped theories of  the testing effect \citep{KarpEtal14}, directed forgetting \citep{SahaEtal13}, retrieval induced forgetting \citep{KlieBaum16}, childhood development \citep{JarroEtal15}, cognitive aging \citep{WahlHuff15,HealKaha15}, event segmentation \citep{EzzyDava14}, time estimation \citep{SahaSmit13}, and even perception \citep{TurkEtal12}. Yet, little is known about how encoding control processes influence contiguity \citep{Hint16}.


Since control processes are deliberately engaged during encoding, their effect can be studied by comparing subjects who have effortfully memorized items to subjects who have only memorized items incidentally with no impetus to engage control processes. Some theories of the TCE suggest task-general processes, that operate both in and outside the laboratory when we unintentionally form new memories, produce the effect \citep{HealEtal14,LohnEtal14}. Such theories predict the TCE should still be observed under incidental encoding. Other theories suggest that the TCE is entirely due to task-specific strategies, implemented by control processes to handle the idiosyncratic demands of laboratory tasks. Such theories predict the TCE should disappear under incidental encoding \citep{Hint16}.


These predictions have never been tested. The only study to measure the TCE after incidental encoding \citep{NairEtal17} did not include an explicit encoding control condition because the TCE was not its main focus. Tantalizingly, this study did \emph{not} find a significant TCE, but the lack of a control condition leaves the reason unclear. Therefore, Experiment 1 directly compares the TCE under explicit and incidental encoding.

\section{Experiment 1}

\section{Method}

% setup some commands to make changing text from study to study easy
\newcommand\listlength{16} % words per list 
\newcommand\presrate{4 seconds} % time per word
\newcommand\isi{1 second} %
\newcommand\DFRDelay{16 seconds} % length of distractor at end of study
\newcommand\recalltime{75 seconds} % time to recall
\newcommand\totalss{XX}
\newcommand\totalexcluded{XX}

\subsection{Subjects}

\subsection{Data Sharing}All data analyzed in this report are freely available on the author's website (cbcc.psy.msu.edu/data/UNIQUELINK).

Given that manipulating encoding intention might reduce, but not eliminate, the TCE, it is critical to have sufficient power to detect small effects. \citet{SedeEtal10} reported a meta-analysis of the TCE in explicit encoding studies; power calculations reveal that a sample size of 143 per condition would provide a $1-\beta$ power of 0.95 to detect (via a 1-tailed 1-sample t-test) an effect one fifth the size of the average effect they reported. 
% here is how to get the SD: ((.614-.5)/31.2)*np.sqrt(510)
In order to collect enough data to meet or exceed this sample size in the Incidental conditions, subjects for all of the experiments reported here were recruited using Amazon's Mechanical Turk, a crowdsourcing website that allows for efficient collection of large volumes of high-quality data. Subjects were paid \$1.00 for participating (a rate of roughly \$10 / hr).

Subjects in the Incidental conditions were excluded from analysis if they reported in a post-experiment questionnaire that they suspected their memory would be tested while they were preforming the judgment task. The final analyzed sample was composed of \shoeExplicitIncluded~in the Explicit condition and \shoeIncidentalIncluded~in the Incidental condition. Table~\ref{sampsize_table} shows the total number of included and excluded subjects for each experiment.
 





\subsection{Procedure}
All subjects completed two free recall lists (only the first of which is analyzed here). Each list was composed of \listlength~words drawn randomly from a pool of 1638 words, with the constraint that no word was used more than once for a given subject. Words were presented one at a time on the subject's computer screen for \presrate. 
%This presentation rate was deliberately chosen to be slightly faster than the 5 seconds per word used by \citet{NairEtal17} to reduce the amount of ``free time'' subjects have between making the judgment and the presentation of the next word.
There was an inter-stimulus interval of \isi~between word presentations during which a fixation cross was displayed in the same location in which the words appeared. The final word of each list was followed by a \DFRDelay~distractor period during which subjects answered math problems of the form $A+B+C=$?, where $A$, $B$, and $C$ were positive, single-digit integers, though the answer could have been one or two digits. Subjects typed their answers to the math problems in a text-box and pressed enter to submit. Upon pressing enter, a new math problem and a new blank text-box appeared. Subjects were instructed to ``Try to solve as many problems as you can without sacrificing accuracy. The task will automatically advance when the time is up.''.

Following the math distractor task, subjects in both conditions were asked to recall as many items as possible from the preceding list, in any order. Subjects typed each recalled word in a text-box and pressed enter to submit the word. Upon pressing enter, the word disappeared and a new blank text-box appeared such that subjects could not see their prior responses. Subjects were given \recalltime~to recall as many words as they could. To ensure subjects noticed that the recall period had begun (e.g., were not looking at the keyboard and typing their answer from the final math problem), a red screen was flashed for 500 ms before the recall instructions were displayed and the recall text-box did not begin accepting input for a further 500 ms. Therefore, including the math distractor, there was a total delay of $16+0.5+0.5=17$ seconds between the end of the study period and the beginning of the recall period. 

% sample size table
\begin{table}
\caption{Sample sizes, exclusions, and recall probability by condition.}
\label{sampsize_table}
\begin{tabular}{llcccc}
\thickline
    Exp & Condition & $n$ Included & $n$ Excluded  & Recall  \\
     &  &  &  (aware) & Prob. (SD) \\
  Exp1  \\
  & Explicit &  \shoeExplicitIncluded & \shoeExplicitAware & \shoeExplicitPrec \\
  & Incidental &  \shoeIncidentalIncluded & \shoeIncidentalAware & \shoeIncidentalPrec \\
    Exp2  \\
  & Explicit &  \doorExplicitIncluded & \doorExplicitAware & \doorExplicitPrec \\
  & Incidental &  \doorIncidentalIncluded & \doorIncidentalAware & \doorIncidentalPrec \\
  Exp3  \\
  & Weight &  \WeightIncluded & \WeightAware & \WeightPrec \\
  & Animacy &  \AnimacyIncluded & \AnimacyAware & \AnimacyPrec \\
  & Scenario &  \ScenarioIncluded & \ScenarioAware & \ScenarioPrec \\
  & Movie  &  \MovieIncluded & \MovieAware & \MoviePrec \\
  & Relational &  \RelationalIncluded & \RelationalAware & \RelationalPrec \\
  
\hline
\end{tabular}
\end{table}

\subsubsection{Encoding Instructions Manipulation} Subjects were randomly assigned to either the Incidental condition or the Explicit condition. Prior to seeing the first list, subjects in both conditions were told that they would see a series of words and would make a simple judgment about each one (i.e., Would it fit in a shoebox?). The exact instructions depended on the condition. In the Explicit condition, subjects were given standard free recall instructions that described the size judgment task but emphasized memory. Because the wording of the instructions are integral to the intent manipulation, they are quoted directly here:

\textbf{Explicit Instructions:}

\begin{displayquote}
        Thank you for participating in this study. 

        We are interested in how people make simple judgments about common words and
        how they subsequently remember the words. Please position this window in the center
        of your screen so you can comfortably view the words.

        You will see a list of words appear one at a time and make a judgment about each one
        (more details on the next page). After the list of words, you will do a few math problems.
        After the math problems, you will be prompted to type in any words that you can remember
        from the list.

        When prompted, type any words that you can remember from the list you just saw,
        \emph{in any order} (type one word in each of the provided text boxes).

    [\textit{a screen showing task instructions, which did not mention memory and was identical for both conditions}]

        Your main task is to remember as many of the words as possible; at the end of the list you will be prompted to
    type as many words as you can remember from the list you just saw, \emph{in any order}.
\end{displayquote}


\textbf{Incidental Instructions:}

\begin{displayquote}
        Thank you for participating in this study.

        We are interested in how people make simple judgments about common words.
        Please position this window in the center of your screen so you can comfortably view the words.

        You will see a list of words appear one at a time and make a judgment about each one
        (more details on the next page). After the list of words, you will do a few math problems.
        
        [\textit{a screen showing task instructions, which did not mention memory and was identical for both conditions}]

        Your main task is to make as accurate a judgment as possible about each word.
\end{displayquote}

\subsubsection{Shoebox Task} In both conditions subjects were asked to make a size judgment about each word while it was present on the screen. Specifically, they were asked  to judge if the word referred to an object that would fit into a shoebox. To allow for the same yes/no response across all the encoding tasks used in all Experiments, subjects were told that we were norming the words for a later study to find items that were neither too easy nor too hard to process and they should press ``Y'' if it was easy to judge if the item would fit in a shoebox and ``N'' if it was difficult to make the judgment. See the supplemental materials for the exact task instructions.

Because subjects completed the task online and could not ask an experimenter for clarification, several measures were taken to ensure that subjects understood how to make a response and could be confident that their responses were being registered: During presentation of the lists, a task prompt was displayed above each word (i.e.., ``Is it easy to judge if it would it fit in a shoebox?''). An instruction about how to make a response was displayed below each word (i.e., ``press ``Y'' for yes, ``N'' for no''). The task prompt and response instructions were in lighter gray font than the black font used for words. The prompts disappeared once the subject made a valid response, but if the subject made an invalid response (e.g., pressing ``B'' instead of ``Y'' or ``N'') the response instructions were replaced with an error message in red font until a valid response was made. The word remained on screen for the full 4 second presentation period regardless of the subject's response.

\subsection{Recall Scoring}
Because subjects typed their responses, typos are likely, and counting only exact matches with list words as correct would underestimate their recall scores. Therefore, a typo-sensitive scoring algorithm was implemented as follows: First, subjects responses were converted to lower case and stripped of any white space. Next, each response was compared to all the list-words that the subject had been presented with up to and including the current list (which were also lowercase and free of white space). If the response exactly matched any of these presented words, it was scored as a correct recall or a prior-list intrusion, depending on whether the matching list-word was presented on the current or a previous list. If the response did not exactly match a presented word, it was compared with each of the 235886 words in Webster's Second International dictionary (https://libraries.io/npm/web2a). If the response exactly matched any word in the dictionary, it was scored as an extra-list intrusion. If the response did not exactly match any word in the dictionary, it was assumed to be a typo and an attempt was made to correct its spelling.

The spell-checking algorithm began by computing the Damerau-Levenshtein distance \citep{Dame64} between the response and each word in the dictionary, providing a measure of the response's similarity to each candidate word. Because almost all responses in free recall correspond to words that were presented on some list (i.e., extra-list intrusions are rare), the algorithm did not automatically replace the mistyped response with the most similar word in the dictionary. Instead it found the shortest distance between the response and an actually presented list-word, and then found where this ``nearest list-neighbor'' distance lay in the distribution of distances between the response and the dictionary words. If the nearest list-neighbor distance was below the tenth percentile of the distribution (i.e., if the response was closer to a list item than it was to 90\% of the words in the dictionary) it was assumed to be that list item, otherwise it is assumed to be an extra-list intrusion.

\subsection{Quantifying the Temporal Contiguity Effect} The TCE is most often examined using a \textit{lag conditional-response probability} function or lag-CRP. The lag-CRP gives the probability that recall of an item studied in position $i$ of a study list will be followed by recall of an item studied in position $i+lag$. For example, if recall of the item from position 5 was followed by recall of the item from position 6 the lag would be 1. If, however, it was followed by recall of the item from position 3, the lag would be -2. For each lag, the CRP is computed by dividing the number of times a transition of that lag was \emph{actually} made by the number of times it \emph{could} have been made \citep[e.g., it could not have been made if the item $i+lag$ was already recalled;][]{Kaha96}. The lag-CRP is typically highest for lags 1 and -1 and decreases sharply for larger absolute values of lag. If the TCE is reduced under incidental encoding conditions, the lag-CRP should be flatter.

The lag-CRP provides a visual representation of the TCE, but it is useful to have a single number that quantifies the size of the effect. For this purpose, the \emph{temporal factor score} is typically used \citep{SedeEtal10,PolyEtal09}. The temporal factor score is computed by ranking the absolute value of the lag of each actual transition with respect to the absolute values of the lags of all transitions that were possible at that time, which provides a percentile score for each transition. Averaging these percentile scores across all of a subject's transitions provides the temporal factor score.
 
When evaluating the size of the TCE, it is important to take into account the fact that some items are more likely to be recalled than others and that the likelihood of successful recall is not random with respect to serial position (e.g., primacy and recency effects, or more generally autocorrelations in goodness of encoding). As a consequence, temporally adjacent items will tend to have similarly high or similarly low probabilities of being recalled, creating temporally isolated ``pockets'' of recallable items, which can influence the size of the TCE \citep{Hint16,SedeEtal10}. For example, if items from the beginning of the list are more likely to be successfully encoded than items from the middle of the list (i.e., a primacy effect), short-lag transitions between primacy items will naturally be more likely than long-lag transitions between primacy and mid-list items (because you can only transition between items that have been encoded), artificially increasing the TCE. In other words, even if recall involved randomly selecting from the pool of successfully encoded items, completely ignoring lag, one would expect a small TCE.

The size of this artificial TCE can be estimated by taking the items which a subject actually recalled for a given list, randomly shuffling (i.e., permuting) the order of recalls, and recomputing the temporal factor score. Repeating this permutation procedure many times provides a distribution of the temporal factor score expected if recall transitions are completely random with respect to lag. This logic was used to provide a corrected measure of the TCE for each participant. For each list, the temporal factor score was computed for the actual recall sequence and for 10,000 random permutations of the sequence.
%(or the maximum number of unique permutations, whichever was smaller). 
The actual temporal factor score was then converted into a z-score, Z(TCE), by subtracting the permutation distribution's mean and dividing by its standard deviation. In the absence of a true TCE, the expected value of Z(TCE) is zero, so we can test for a TCE by determining if the across-subject average of Z(TCE) is significantly above zero.



\section{Results}

\newcommand\paneltext{(A) Lag-conditional response probability functions. Error bars are bootstrapped within-subject 95\% confidence intervals. (B) The average Z(TCE).  Error bars are bootstrapped between-subject 95\% confidence intervals. Z(TCE) for a given subject is computed as follows: An observed temporal factor score was computed as the average percentile ranking the temporal lag of each actual transition in the recall sequence with respect to the lags of all transitions that were possible at that time. To determine the temporal factor score expected by chance, a permutation distribution was created by randomly shuffling the order of recalls within the sequence 10,000 times and computing a temporal factor score for each shuffling. The reported value, Z(TCE), is z-score of the observed temporal factor score within the permutation distribution.}
\begin{figure}
\fitfigure{figures/shoebox.pdf}
\caption{The temporal contiguity effect (TCE) with the Shoebox size judgment task under explicit versus incidental encoding. \paneltext}
\label{shoebox}
\end{figure}



Figure~\ref{shoebox} shows the lag-CRP and corrected temporal factor scores for the Explicit and Incidental conditions. The Explicit condition shows a clear TCE: the lag-CRP is highest for short lags (i.e., $|lag|=1$) and decreases for larger lags. Moreover, the 95\% confidence interval on the Z(TCE) lies well above zero. By contrast, the Incidental condition shows no evidence of a TCE: the lag-CRP is nearly flat and the 95\% confidence interval on the Z(TCE) includes zero. Overall recall probability (Table~\ref{sampsize_table}) was also lower in the Incidental condition.

These results suggest that removing the intent to encode eliminates the TCE. However, because the TCE has proven to be so robust, one can be justifiably skeptical of a single experiment showing lack of contiguity. Therefore, we attempt to replicate the finding in Experiment 2 using a slightly different processing task.

\section{Experiment 2}
\section{Method}

The methods were identical to those used in Experiment 1 except the judgment task instructions (see Table 1 for sample size information).

The processing required by the Shoebox Task from Experiment 1 is quite simple. So simple that one could argue it ineffective at forming strong memories, which may artificially reduce the TCE. Therefore, we wanted to retain the basic task of judging size while increasing memory performance in the Incidental condition. That is, can processing that promotes memory do so without producing substantial contiguity? Mental imagery and self-referential processing are two effective ways to improve memory. Thus, the Front Door Judgment task asked subjects to imagine trying to move the object referred to by each item through the front door of their house and decide whether or not it would be possible (again, subjects were asked to indicate if this judgment was easy or difficult to make by pressing ``Y'' or ``N'').
%This Front Door task, should encourage subjects to form a more vivid and self-relevant mental image than the Shoebox task.
See the supplemental materials for the exact task instructions.

\section{Results}
As predicted, the Front Door Task substantially improved memory accuracy in the Incidental condition to approximately the level seen in the Explicit condition of Experiment 1 (Table 1). The Front Door task did not, however, produce a significant TCE. Figure~\ref{door} shows that whereas the Explicit condition showed a distinctly peaked lag-CRP and a Z(TCE) significantly above zero, the Incidental condition showed a flattened lag-CRP and a Z(TCE) for which the confidence interval included zero.

\begin{figure}%[hp]
\fitfigure{figures/FrontDoor.pdf}
\caption{The temporal contiguity effect (TCE) with the Front Door size judgment task under explicit versus incidental encoding. \paneltext}
\label{door}
\end{figure}

\section{Interim Discussion}
Experiments 1 and 2 show that the TCE can be absent when intent to encode is absent. This result is consistent with theories that ascribe the TCE to strategic control processes. Under this interpretation, the contiguity-generating processes are more or less inseparable from the intent to encode. But is intent to encode truly necessary to find a TCE? Perhaps not.

An alternate interpretation is that automatic encoding processes do produce contiguity, but their effect is obscured by processes required by the judgment task. For example, most models produce a TCE because the representations of items studied close together are more similar to each other than they are to the representations of items studied far apart. The Shoebox and Front Door tasks encourage subjects to maintain a common mental representation (e.g., image of a shoebox) throughout the list presentation. If this representation is incorporated into the representations of list items, it would increase the similarity of items separated by distant lags, attenuating the TCE. When trying to memorize, subjects likely process items in ways that are not necessry the judgment task, perhaps decreasing the similarity of items seperated by distant lags, increasign the TCE. That is, the judgment task might decrease the TCE in a way that is not due to the lack of intent to encode.

More generally, if only intentional control processes produce contiguity, it should be challenging, perhaps impossible, to observe a TCE under incidental encoding. But if contiguity is simply obscured by some types of task-related processing, it should be easy to find incidental encoding tasks that produce a TCE. Experiment 3 tests these predictions by examining five different encoding tasks.








\section{Experiment 3}
\section{Method}
The question is no longer whether Explicit encoding produces a larger TCE than Incidental encoding, but rather whether the TCE can ever be observed under Incidental encoding. Thus, in Experiment 3 all subjects were given Incidental encoding instructions, but were randomly assigned to one of five different judgment tasks that varied in the type of processing required. Otherwise, the methods were identical to those used in Experiments 1 and 2 (see Table 1 for sample size information).

\subsection{Processing Task Manipulation}
In all conditions, subjects were asked to make a judgment about each word as it was presented. Here, we describe the type of processing that each task was intended to discourage (or encourage). Again, to allow for the same yes/no response for each task, subjects were asked to indicate if the judgment was easy to make under the guise of norming the items for a later study. See the supplemental materials for the exact task instructions.

\subsubsection{Weight Task} The Weight Task was similar to the size judgment tasks used in the first two experiments except that it asked subjects to compare each item's \emph{weight} to a common referent: a bottle of water. Specifically, they were asked to judge whether each word referred to an object that was heavier than ``a standard bottle of water you'd purchase from a vending machine''. Because weight is not an easily visualizable attribute, the Weight Task might be expected to reduce the likelihood that subjects will maintain the same vivid mental image throughout the list. Thus, it may produce a larger TCE if associating each item with a common mental image tends to attenuate the TCE.

\subsubsection{Animacy Task} The Animacy Task asks subjects whether each item refers to an object that is living or non-living. Like the Shoebox, Front Door, and Weight tasks, the Animacy task requires subjects to consider only a single attribute of each item (i.e., animacy status). But unlike the aforementioned tasks, it does not provide a reference object against which to compare each item. Thus, it further reduces the likelihood of maintaining a single vivid image throughout the list. 

\subsubsection{Scenario Task} The Scenario Task asks subjects to judge the relevance of each word to a scenario: moving to a foreign land \citep{NairEtal17}. Subjects are likely to maintain some representation of this scenario across items, but because it does not specify any pre-existing dimension, like size or weight, each item may be expected to activate many different attributes, lowering the similarity of mental representations from item to item.

\subsubsection{Movie Task} The instructions for the Movie Task explain that ``when you read a word, it can trigger many different thoughts'' and gives the example of the word baseball triggering a series of thoughts: ``you might have a mental image of a baseball, you might hear the crack of a bat hitting a ball, you might think of related concepts like ballpark, players, and fans...''. It then asks subjects to allow each item ``to activate as many different thoughts as possible. Then use these thoughts to generate a mental movie (like a detailed image of spending an afternoon at a baseball game or what it is like to be a player on a baseball field).'' Subjects then judge whether or not it was easy to form such a mental movie. This tasks removes the requirement to consider each item along the same dimensions and instead encourages subjects to think deeply about the unique attributes of each item, which might be expected to cause very different mental representations to be activated with each successive item, perhaps increasing the TCE. 

\subsubsection{Relational Task} The Relational task is similar to the Movie Task except instead of being asked to make a new mental movie for each item, subjects are asked to ``try to incorporate each new word into your existing mental movie. For example, if the next word was "owner", you should allow it to activate many associated thoughts and then incorporate it into your existing ``ballpark'' movie.'' By explicitly asking subjects to form links between temporally proximate items, this condition should maximize the chance of observing a TCE. 

\section{Results}

As seen in Figure~\ref{E3}, all of the processing tasks produced a TCE under incidental encoding conditions. For each task, the lag-CRP tends to decrease with increasing $|lag|$ and the Z(TCE) is significantly above zero. These results show that while the TCE can be attenuated under incidental conditions (as in Experiments 1 and 2), the lack of intent to encode, per se, does not eliminate contiguity. 

Indeed, perhaps the most remarkable feature of the data is how little the size of the TCE differs among the tasks, consistent with the suggestion that the TCE is due to automatic encoding processes that are independent of the judgment task. %The only condition for which the z(TCE) differed significantly from any other condition was the Relational Task condition, which asked subjects to integrate each item in to an ongoing movie.
In other words, robust contiguity of approximately the same magnitude was observed regardless of the processing task unless subjects were explicitly encouraged to integrate the current item with past items.

\begin{figure}%[hp]
\fitfigure{figures/E3.pdf}
\caption{The temporal contiguity effect (TCE) under incidental encoding with different judgment tasks. \paneltext}
\label{E3}
\end{figure}

\section{General Discussion}
The source of the TCE has been a matter of debate. Is it due to task-general automatic memory encoding processes that operate whenever new memories are formed \citep{HealEtal14}? Or is the TCE a product of control processes that implement task-specific encoding strategies to handle the idiosyncratic demands of laboratory tasks \citep{Hint16}? The former possibility suggests that the TCE should be observable under almost any encoding circumstances, the latter suggests the TCE should be eaisly eliminated by removing the imptus to engage controled encoding processes. The results suggest that neither view is fully correct.

Experiments 1 and 2 clearly showed that how participants choose to process information at encoding matters: in these experiments, the TCE was eliminated when controled encoding processing was discuraged by incidental encoding instructions. This finding points to a large gap in our understanding of how encoding processes influence contiguity. There has been very little work modeling how automatic memory processes interact with controled processes to meet task demands \citep{LehmMalm13,PolyEtal09}, thus existig models would likely have difficulty accounting for the difference between the Explicit and Incidental conditions. The challenge is avoiding adding a humunclus to the models that does the hard work of inrepreting task instructions and switches on some otherwise dormant process. What computational mechanisms all instructions to regulate encoding processes?

Just as clearly, however, Experiment 3 showed that the TCE does not require intent: a robust TCE was observed in five different implicit encoding tasks. This finding strongly suggest that the TCE is not completley dependent on controled, strategic encoding processes as some theories have suggested \citep{Hint16}. That is, these results are an existence proof for temporal contiguity under incidental encoding and show that the TCE is not due to task-specific strategies implemented by controlled encoding process \citep[cf.][]{Hint16}.

In summary, the results support theories that ascribe the TCE to task-general memory processes. But they also point to serrious limitaitons of existing theories---we understand much about the processes that encoding and search memories, but little about how these processes are controled.






% Some theories of the TCE suggest the effect is produced by task-general memory encoding processes that operate both in laboratory tasks and outside the laboratory when we form new memories without trying to memorize anything \citep{HealEtal14,LohnEtal14}. Such theories predict the TCE should be easily observed under incidental encoding. Other theories suggest that the TCE is entirely due to task-specific strategies implemented by control processes to handle the idiosyncratic demands of laboratory tasks. Such theories predict the TCE should disappear under incidental encoding \citep{Hint16}.






















\bibliography{healey_lab}
\end{document}